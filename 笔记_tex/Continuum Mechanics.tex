\documentclass[UTF8,openany,12pt]{ctexart}
\usepackage{amsmath}
\usepackage[colorlinks,linkcolor=black,anchorcolor=blue,citecolor=green]{hyperref}    %修改特定内容的颜色
\usepackage{enumerate}                      %条目包
\usepackage{float}                          %固定表格位置
\usepackage{graphicx}                       %图片包
\usepackage{float}                          %设置图片浮动位置的宏包
\usepackage{subfigure}                      %插入多图时用子图显示的宏包
\usepackage{titlesec}                       %自定义多级标题格式的宏包
\usepackage{geometry}
\usepackage{wrapfig}
\pagestyle{plain}						    %显示页码
\RequirePackage{fix-cm}                     
\usepackage{longtable}
\usepackage{listings}
\usepackage{setspace}
\usepackage{xcolor}
\usepackage{caption}
\usepackage{diagbox}
\usepackage{multirow} %合并多行单元格的宏包
\usepackage{longtable} %不宽但很长的表格可以用longtable宏包来进行分页显示
\usepackage{array} %一般用于数学公式中对数组或矩阵的排版
\usepackage{makecell}% makecell命令对表格单元格中的数据进行一些变换的控制。我们可以使用 \ 命令进行换行,也可以使用p{(宽度)}选项控制列表的宽度
\usepackage{threeparttable} %制作三线表格
\usepackage{booktabs}%s三线表格中的上中下直线线型设置宏包,在这个包中水平线被教程\toprule、midrule、buttomrule。
\usepackage{tikz}
\usetikzlibrary{arrows.meta, positioning}
\usepackage{tcolorbox}
\usepackage{cancel}
\tcbuselibrary{theorems, skins, breakable}
\newtcolorbox{remark}{
    enhanced,
    breakable,
    colback=gray!5!white,
    colframe=gray!30!white,
    boxrule=0.5pt,
    arc=2pt,
    title= ,
    fonttitle=\bfseries,
    left=10pt,
    right=10pt
}
\newtcolorbox{example}{
    enhanced,
    breakable,
    colback=blue!5!white,
    colframe=blue!25!white,
    boxrule=0.5pt,
    arc=2pt,
    title= 例子,
    fonttitle=\bfseries,
    left=10pt,
    right=10pt
}
\newtcolorbox{proof}{
    enhanced,
    breakable,
    colback=red!5!white,
    colframe=red!25!white,
    boxrule=0.5pt,
    arc=2pt,
    title= 证明,
    fonttitle=\bfseries,
    left=10pt,
    right=10pt
}

%各级标题格式自定义
% \titleformat{\section}[block]{\Large\bfseries}{\arabic{section}\quad }{1em}{}[]
% \titleformat{\subsection}[block]{\large\itshape\bfseries}{\arabic{section}.\arabic{subsection}}{1em}{}[]
% \titleformat{\subsubsection}[block]{\normalsize\bfseries}{\arabic{section}.\arabic{subsection}.\arabic{subsubsection}}{1em}{}[]
% \titleformat{\paragraph}[block]{\small\bfseries}{[\arabic{paragraph}]}{1em}{}[]

\linespread{1.5}

\geometry{a4paper,left=2cm,right=2cm,top=2cm,bottom=2cm}

\title{\textbf{连续介质力学笔记}}
\author{\url{https://github.com/Rrmeazh/Elementary_Continuum_Mechanics}}
\date{\today}

\begin{document}

\maketitle
\tableofcontents
\thispagestyle{empty}

\newpage

\part{运动学与张量运算}

\section{运动的描述}

\subsection{连续介质假设}

顾名思义,连续介质是连续地充满所占空间的物质。这里所说的连续要在宏观意义下理解,因为物质在微观观点下以离散方式存在(例如气体由大量分子组成,晶体由大量原子组成)。在连续介质假设下,我们忽略物质在微观尺度上的离散性,从而认为物质在空间中连续分布。

连续介质假设的前提是在合适的宏观尺度上考虑问题。因为宏观和微观具有相对性,所以连续介质假设也具有相对性。同一种物质,在一些问题中是连续介质,而在另一些问题中不是连续介质,这取决于所考虑问题的尺度。

\begin{remark}
    例如,液态的水通常被视为连续介质。但当我们研究一粒花粉在水面上的布朗运动时,由于花粉的尺度很小,水与花粉之间的相互作用不能简单地看做连续分布的质量与微粒之间的相互作用。

    类似地,空气通常是连续介质。但非常稀薄的空气(例如大气层顶部的空气)在人造飞行器的尺度上会表现出不同寻常的力学性质。不能简单地在连续介质力学的范围内进行研究。
\end{remark}
 
要想检验连续介质假设是否成立,就要看基本粒子的涨落现象在我们所关心的尺度上是否表现出来。

根据微积分的基本思想,尤其是连续函数的概念,连续介质假设具有彼此等价的两方面含义:

\begin{itemize}
    \item 一方面,如果把所研究的物质分割为诸多微元,则当分割越来越细并取极限时,各物质微元分别化为相应的物质点。而各物质微元的平均特征量分别趋于相应物质点的特征量(如密度、温度、速度等)。于是所研究的物质由在空间中连续分布的物质点组成,每一个物质点都具有确定的不具有随机性的宏观特征量。并且这些特征量是物质点的连续函数。
    \item 另一方面,如果把物质所占空间分割为诸多微元,则当分割越来越细并取极限时,各空间微元分别化为相应的空间点。而各空间微元中的物质的平均特征量分别趋于物质在相应空间点的特征量。于是得到所研究物质在各空间点确定的不具有随机性的值,并且这些值是空间点的连续函数。
\end{itemize}
 
连续介质假设的这两方面含义对应着描述连续介质运动的两种观点(拉格朗日观点与欧拉观点)。

\begin{remark}
    注意:连续介质的物质点不同于理论力学中的质点,因为通常认为质点离散分布并且具有有限的质量。为了避免歧义,我们不把连续介质的物质点简称为质点。
\end{remark}

总之,可以用连续介质力学方法进行研究的物质对象应当由诸多微元组成,这些微元的特征可由某种平均意义下的宏观参量来描述,并且:

\begin{itemize}
    \item 微元的尺度一方面远大于分子动理学尺度(指分子的尺度、相邻分子之间的距离、分子平均自由程、品格观上够大尺度等)或对象本身的不均匀性尺度(例如雪崩中一片雪花的尺度和相邻雪花之间的距离、多孔介质中的孔隙尺度),另一方面远小于描述该对象的宏观参量在空间中发生显著变化的尺度(例如运动区域的特征尺度、运动本身的特征尺度、波长)。
    \item 换言之,连续介质微元在微观上应当足够大,在宏观上应当足够小。
\end{itemize}

当这些条件成立时,这些微元在极限意义下给出连续介质的物质点,并且在求极限时已经不再考虑分子动理学尺度或对象本身不均匀性尺度上的离散性,从而认为物质在空间中连续分布。类似地,在极限意义下也可以得到连续介质各宏观参量在空间中的分布。

通过对时间尺度(宏观变化的特征时间与基本粒子相互作用的特征时间)的类似分析可知,在连续介质假设下,各宏观参量对时间的依赖关系也应当是连续的。需要强调的是,在提出连续介质假设时应当明确定义各宏观参量。用来描述连续介质运动的宏观参量一般指那些能够反映介质微元状态的量,例如速度、密度、温度、压强等。这些量在空间中的分布给出相应的场。反映连续介质某种整体性质的量,例如连续介质的总质量、通过某截面的通量等,并不是本书所说的宏观参量,而是相应宏观参量的某种积分。

固体和流体(液体和气体的统称)是典型的连续介质,而刚体是最简单的连续介质(各物质点之间的距离永远保持不变)。从另一个角度说,溶液、合金、空气等均相混合物也是典型的连续介质,而大量非均相混合物,如胶体、悬浊液、乳状液等以及各种复合材料、多孔介质甚至粉末状物质等在一定精度下也是连续介质力学的研究对象。

连续介质假设的精度与所研究问题本身的精度应当彼此匹配,在一定精度下提出连续介质假设后,关心该精度以下的细节是没有意义的。

\begin{remark}
    例如,为了利用计算流体力学方法预报天气,我们认为空气连同其中的云雾、雨雪甚至沙尘合在一起是连续介质,而天气预报所关心的尺度在水平方向达到 $10km$ 甚至 $100km$ 以上,在竖直方向一般也在 $1km$ 以上,并且空气的水平运动通常显著强于竖直运动,所以一座高楼的影响可以忽略不计,但一座高山可能对局部天气产生显著影响。
    
    在提出边界条件时,必须在适当精度下考虑山脉的起伏,而一般不必考虑城市轮廓和地表建筑细节。不过,由于城市的能源消耗越来越大,有时需要考虑城市的热效应对局部天气的影响。
\end{remark}


连续介质假设是应用微积分工具来研究连续介质运动的基本前提之一,它要求描述连续介质状态的物理量连续分布。当然,仅仅要求连续分布是不够的,我们往往还额外要求相关函数光滑(即其一阶导数存在并且连续)或分段光滑,而介质本身的拓扑结构应保持不变。这些附加假设其实是微积分中的几个常用公式成立的前提条件。

在连续介质力学的范围内也可以研究一些带有间断面的问题,例如气体中的激波和水面上的波动。一般而言,只要间断面的数量是有限的,我们就可以在间断面上提出合适的边界条件,并用连续介质力学方法研究间断面以外介质的运动以及间断面本身的运动。

连续介质是大量实际介质的理想化模型。提出连续介质假设是建立连续介质模型的第一步,后面还需要解决大量问题,包括描述连续介质的运动、建立连续介质的封闭方程组等。为此,我们还要提出许多假设,连续介质模型是否成立,归根结底要通过实验来验证。

\subsection{拉格朗日观点}

\begin{remark}
    物质体是指由一些物质点组成的几何体,其表面称为物质面。物质面也指由同样一些物质点组成的曲面(不一定是封闭的)。物质线的概念是类似的。物质体、物质面和物质线都随着连续介质一起运动。
\end{remark}

拉格朗日观点描述连续介质各物质点的物理量。为区分不同的物质点,用有序的三个数 $\xi^1, \xi^2, \xi^3$ 表示物质点,这三个数称为该物质点的拉格朗日坐标(也称为随体坐标或物质坐标)。一般采用物质点在初始时刻 $t_0$ 的坐标为拉格朗日坐标。此时连续介质的坐标由以下函数给出:

$$
x^i = x^i(\xi^1, \xi^2, \xi^3, t) \qquad i = 1, 2, 3
$$

即:

$$
\boldsymbol{r} = \boldsymbol{r}(\xi^1, \xi^2, \xi^3, t)
$$

显然初始位置由以下函数给出:

$$
\mathring{\boldsymbol{r}} = \boldsymbol{r}(\xi^1, \xi^2, \xi^3, t_0)
$$

物质点的速度、加速度分别为:

$$
\boldsymbol{v}(\xi^1, \xi^2, \xi^3, t)
= \frac{\partial \boldsymbol{r}(\xi^1, \xi^2, \xi^3, t)}{\partial t}
$$

$$
\boldsymbol{a}(\xi^1, \xi^2, \xi^3, t)
= \frac{\partial \boldsymbol{v}(\xi^1, \xi^2, \xi^3, t)}{\partial t}
= \frac{\partial^2 \boldsymbol{r}(\xi^1, \xi^2, \xi^3, t)}{\partial t^2}
$$

一般而言,物理量 $A$ 在拉格朗日坐标不变的条件下对时间的偏导数 $(\frac{\partial A}{\partial t})_{\xi^i} = \frac{\partial A(\xi^1, \xi^2, \xi^3, t)}{\partial t}$ 称为该物理量的物质导数或随体导数。它是拉格朗日坐标为 $\xi^1, \xi^2, \xi^3$ 的同一物质点的物理量 $A$ 对时间的变化率。速度是径矢/位移的物质导数,加速度是速度的物质导数。

若出现物质点的合并、分开,其一般只在线、面局部发生,此时可以引入新的拉格朗日坐标。

\subsection{欧拉观点}

\begin{remark}
    控制体是指在给定空间中按照已知规律运动的几何体,其表面称为控制面,以后经常使用静止的控制体和控制面
\end{remark}

欧拉观点描述连续介质各空间点的物理量。例如,如果从上面的 $x^i = x^i(\xi^1, \xi^2, \xi^3, t)$ 反解出:

$$
\xi^i = \xi^i(x^1, x^2, x^3, t) \qquad i = 1, 2, 3
$$

即可知道时刻 $t$ 位于任一空间点的物质点究竟是哪一个。这种信息往往过于细致,一般我们更关心空间某一点的速度、密度等信息。

\begin{remark}
    需要注意的是,流线与迹线是不同的概念。
\end{remark}

为简洁起见,我们在之后出现的求和表达式中省略求和符号,并约定同样的一个上标和一个下标表示对相应的角标从1到3(在二维空间中从1到2)求和。这样的约定称为求和约定。

表示求和的一对角标称为哑标或傀标。哑标可以成对地改为在相应表达式中没有出现的任何其他角标。需要特别强调,成对出现的两个上标或两个下标都不表示求和(其他文献中只使用下标的情况另当别论)。

此外,我们认为偏导数记号 $\frac{\partial}{\partial x^i}$ 中的角标 $i$ 相当于下标(在文献中也广泛使用记号 $\partial_i$ 来表示对坐标 $x^i$ 的偏导数),以便省略求和符号。

\begin{example}
    \begin{itemize}
        \item $a^{ij} b_{ij}$ 表示 $\sum_{i = 1}^3 \sum_{j = 1}^3 a^{ij} b_{ij}$
        \item $\delta_i^i$ 表示 $\sum_{i = 1}^3 \delta_i^i$
        \item $\frac{\partial v^i}{\partial x^i}$ 表示 $\sum_{i = 1}^3 \frac{\partial v^i}{\partial x^i}$
    \end{itemize}
\end{example}

考虑在欧拉观点下计算物理量 $A = A(x^1, x^2, x^3, t)$ 的物质导数的问题。一种显而易见的方法是代入 $x^i = x^i(\xi^1, \xi^2, \xi^3, t)$ 从欧拉观点转化到拉格朗日观点。利用复合函数求导公式得:

$$
(\frac{\partial A}{\partial t})_{\xi^i}
= \frac{\partial A(x^1, x^2, x^3, t)}{\partial t} + \frac{\partial A(x^1, x^2, x^3, t)}{\partial x^j} (\frac{\partial x^j}{\partial t})_{\xi^i}
$$

其中 $(\frac{\partial x^j}{\partial t})_{\xi^i}$ 其实是速度的一种分量 $v^j$ ,在2.2节将进一步解释。在欧拉观点中,可用记号 $\frac{d}{dt}$ 或 $\frac{D}{Dt}$ 表示物质导数,于是:

$$
\frac{dA}{dt}
= \frac{\partial A}{\partial t} + v^j \frac{\partial A}{\partial x^j}
$$

上式右边的两项分别称为 $A$ 的局部导数和对流导数。

局部导数与 $A$ 随时间的变化有关。不随时间变化的场称为定常场,随时间变化的场称为非定常场。定常场的局部导数为零。

对流导数也称为位变导数或迁移导数,它不但与场随坐标的变化有关,而且与速度场 $v$ 有关。不随坐标变化的场称为均匀场,随坐标变化的场称为非均匀场。均匀场的对流导数为零,但对流导数为零的情况并非仅限于均匀场。

\begin{remark}
    由此可见,为了在欧拉观点下计算物质导数:一般还必须知道速度场。
\end{remark}

既不随时间变化,也不随坐标变化的场称为恒定场。

此外,也可以直接推导物质导数公式。设物质点在时刻 $t$ 位于空间点 $\boldsymbol{r}$ ,在时刻 $t + \Delta t$ 位于空间点 $\boldsymbol{r} + \Delta \boldsymbol{r}$ ,则:

$$
\frac{dA}{dt}
= \lim_{\Delta t \to 0} \frac{A(\boldsymbol{r} + \Delta \boldsymbol{r}, t + \Delta t) - A(\boldsymbol{r}, t)}{\Delta t}
$$

可将上式对 $\Delta x^i$ 和 $\Delta t$ 展开,也可做如下操作:

$$
\frac{dA}{dt}
= \lim_{\Delta t \to 0} \frac{A(\boldsymbol{r} + \Delta \boldsymbol{r}, t + \Delta t) - A(\boldsymbol{r} + \Delta \boldsymbol{r}, t)}{\Delta t} + \lim_{\Delta t \to 0} \frac{A(\boldsymbol{r} + \Delta \boldsymbol{r}, t) - A(\boldsymbol{r}, t)}{\Delta t}
$$

上式右边第一项即为局部导数,第二项等于

$$
\lim_{\Delta t \to 0} \frac{|\Delta \boldsymbol{r}|}{\Delta t} \lim_{\Delta t \to 0} \frac{A(\boldsymbol{r} + \Delta \boldsymbol{r}, t) - A(\boldsymbol{r}, t)}{|\Delta \boldsymbol{r}|}
= v \frac{\partial A}{\partial v_0}
$$

其中 $v$ 为速度大小, $\frac{\partial A}{\partial v_0}$ 为 $A$ 沿速度方向上的单位矢量 $v_0$ 的方向导数。

\section{坐标系及其变换}

\subsection{基矢量}

给定曲线坐标系 $x^1, x^2, x^3$ ,点的径矢就可以用 $\boldsymbol{r} = \boldsymbol{r}(x^1, x^2, x^3)$ 表示。在每一个给定的点 $M$ 可以引三条坐标线,通过点 $M$ 的坐标线 $x^i$ 是第 $i$ 个坐标取所有可能的值而其余坐标与 $M$ 的相应坐标分别相同的点的集合,而 $x^i$ 增加的方向给出坐标线 $x^i$ 的定向。由此自然可以定义基矢量 $\boldsymbol{e}_i$:

$$
\boldsymbol{e}_i
= \frac{\partial \boldsymbol{r}}{\partial x^i}
$$

其方向与坐标线 $x^i$ 在点 $M$ 的切线方向一致,且指向坐标 $x^i$ 增加的方向。

\begin{example}
    柱面坐标系 $x^1 = r, x^2 = \theta, x^3 = z$ 的基矢量:
    \begin{itemize}
        \item $\boldsymbol{e}_1$ 是沿坐标线 $r$ (径向)的单位矢量
        \item $\boldsymbol{e}_2$ 是沿坐标线 $\theta$ (圆周逆时针切向)且长度为 $r$ 的矢量
        \item $\boldsymbol{e}_3$ 是沿坐标线 $z$ (轴向)的单位矢量
    \end{itemize}
\end{example}

\subsection{径矢微分的分解式}

点 $M$ 处的径矢可以按该点处的基矢量分解:

$$
d\boldsymbol{r}
= \frac{\partial \boldsymbol{r}}{\partial x^i} dx^i
= \boldsymbol{e}_i \, dx^i
$$

利用此分解式可以解释欧拉坐标 $x^i$ 的物质导数的含义。由速度的定义:

$$
\boldsymbol{v}
= (\frac{\partial \boldsymbol{r}}{\partial t})_{\xi^i}
= (\frac{\partial x^j}{\partial t})_{\xi^i} \, \boldsymbol{e}_j
$$

所以 $x^i$ 的物质导数就是速度关于基矢量的分解式 $\boldsymbol{v} = v^j \boldsymbol{e}_j$ 中的系数 $v^j$ ,即速度分量

$$
v^j = (\frac{\partial x^j}{\partial t})_{\xi^i}
$$

\subsection{坐标变换}

考虑两个坐标系 $x^1, x^2, x^3$ (称为旧坐标系)和 $x'^1, x'^2, x'^3$ (称为新坐标系),其基矢量分别为 $\boldsymbol{e}_1, \boldsymbol{e}_2, \boldsymbol{e}_3$ 和 $\boldsymbol{e}'_1, \boldsymbol{e}'_2, \boldsymbol{e}'_3$ ,设他们之间的变换关系由已知的可微函数给出:

$$
x'^i = x'^i(x^1, x^2, x^3) \quad x^i = x^i(x'^1, x'^2, x'^3)
$$

显然径矢微分的分解式在这两个系中形式相同,即:

$$
d\boldsymbol{r}
= \boldsymbol{e}_i \, dx^i
= \boldsymbol{e}'_i \, dx'^i
$$

下面研究 $dx^i$ 和 $\boldsymbol{e}_i$ 的变换规律

取函数 $x'^i = x'^i(x^1, x^2, x^3)$ 的微分,得到:

$$
dx'^i = \frac{\partial x'^i}{\partial x^j} dx^j
$$

\begin{remark}
    在任意坐标变换中都满足这种变换方式的量称为逆变的量(一阶情况)
\end{remark}

上式也可写成矩阵形式,记从坐标 $x^i$ 到坐标 $x'^i$ 的雅可比矩阵

$$
\boldsymbol{J}
= (\frac{\partial x'^i}{\partial x^j})
= \begin{pmatrix}
    \frac{\partial x'^1}{\partial x^1} & \frac{\partial x'^1}{\partial x^2} & \frac{\partial x'^1}{\partial x^3} \\
    \frac{\partial x'^2}{\partial x^1} & \frac{\partial x'^2}{\partial x^2} & \frac{\partial x'^2}{\partial x^3} \\
    \frac{\partial x'^3}{\partial x^1} & \frac{\partial x'^3}{\partial x^2} & \frac{\partial x'^3}{\partial x^3} \\
\end{pmatrix}
$$

则:

$$
\begin{pmatrix}
    dx'^1 \\
    dx'^2 \\
    dx'^3 \\
\end{pmatrix}
= \boldsymbol{J}
\begin{pmatrix}
    dx^1 \\
    dx^2 \\
    dx^3 \\
\end{pmatrix}
$$

或

$$
(dx'^1,dx'^2,dx'^3)
= (dx^1,dx^2,dx^3) \boldsymbol{J}^T
$$

再考虑基矢量,显然:

$$
\boldsymbol{e}'_i
= \frac{\partial \boldsymbol{r}}{\partial x'^i}
= \frac{\partial x^j}{\partial x'^i} \frac{\partial \boldsymbol{r}}{\partial x^j}
= \frac{\partial x^j}{\partial x'^i} \boldsymbol{e}_j
$$

\begin{remark}
    称这种变换为协变的
\end{remark}

矩阵形式为

$$
\begin{pmatrix}
    \boldsymbol{e}'_1 \\
    \boldsymbol{e}'_2 \\
    \boldsymbol{e}'_3 \\
\end{pmatrix}
= \boldsymbol{J}^{-1T}
\begin{pmatrix}
    \boldsymbol{e}_1 \\
    \boldsymbol{e}_2 \\
    \boldsymbol{e}_3 \\
\end{pmatrix}
$$

或

$$
(\boldsymbol{e}'_1,\boldsymbol{e}'_2,\boldsymbol{e}'_3)
= (\boldsymbol{e}_1,\boldsymbol{e}_2,\boldsymbol{e}_3) \boldsymbol{J}^{-1}
$$

\begin{example}
    用基矢量的变换式计算柱面坐标系的基矢量

    直角坐标系 $x^1 = x, x^2 = y, x^3 = z$ 与柱面坐标系 $x'^1 = r, x'^2 = \theta, x'^3 = z$ 之间的关系为

    \begin{align*}
        x & = r \cos \theta \\
        y & = r \sin \theta \\
        z & = z
    \end{align*}

    故:

    $$
    \begin{pmatrix}
        \boldsymbol{e}'_1 \\
        \boldsymbol{e}'_2 \\
        \boldsymbol{e}'_3 \\
    \end{pmatrix}
    = \boldsymbol{J}^{-1T}
    \begin{pmatrix}
        \boldsymbol{e}_1 \\
        \boldsymbol{e}_2 \\
        \boldsymbol{e}_3 \\
    \end{pmatrix}
    = \begin{pmatrix}
        \cos \theta & \sin \theta & 0 \\
        -r \sin \theta & r \cos \theta & 0 \\
        0 & 0 & 1
    \end{pmatrix}
    \begin{pmatrix}
        \boldsymbol{e}_1 \\
        \boldsymbol{e}_2 \\
        \boldsymbol{e}_3 \\
    \end{pmatrix}
    $$
\end{example}


\section{张量的概念}

\subsection{一阶张量(矢量)及其并积}

如果一个量可以按照基矢量分解,并且相应系数在坐标变换时按照逆变公式变换,则这样的量称为一阶张量,而这些系数称为其逆变分量。一阶张量通常也称为矢量。径矢、位移、速度、加速度等都是矢量。

\begin{remark}
    基矢量本身不满足矢量的以上定义,所以基矢量不是矢量。这个类似于白马非马的结论看似荒谬,其实源于对矢量概念的不同理解。(广义的)矢量经常被理解为既有大小也有方向的数学对象,可以用有向线段表示,而基矢量正是通过这样的矢量定义出来的。基矢量与坐标系有关,是坐标系的属性。(狭义的)矢量定义则要求矢量在按照基矢量分解时具有特定结构,这种结构保证了矢量的大小和方向与所采用的坐标系无关。
\end{remark}

矢量相当于基矢量的一种线性组合,相应系数是逆变的,从而保证这种线性组合在坐标变换下具有不变性。对此略作推广,就能得到张量的定义。为此需要定义矢量的并积。

把多个矢量按顺序并排写在一起,是矢量的一种乘法运算(并乘),其结果称为并积。例如,矢量 $\boldsymbol{a}$ 与 $\boldsymbol{b}$ 的并积为 $\boldsymbol{ab}$ 。并乘运算满足结合律与分配律。即对任意矢量 $\boldsymbol{a},\boldsymbol{b},\boldsymbol{c}$ 和普通的数 $\lambda,\mu$ ,有:

$$
\boldsymbol{a} \boldsymbol{b} \boldsymbol{c}
=(\boldsymbol{a} \boldsymbol{b}) \boldsymbol{c}
=\boldsymbol{a} (\boldsymbol{b} \boldsymbol{c})
$$

$$
\boldsymbol{a} (\lambda \boldsymbol{b} + \mu \boldsymbol{c})
=\lambda \boldsymbol{a} \boldsymbol{b} + \mu \boldsymbol{a} \boldsymbol{c}
$$

$$
(\lambda \boldsymbol{a} + \mu \boldsymbol{b}) \boldsymbol{c}
=\lambda \boldsymbol{a} \boldsymbol{c} + \mu \boldsymbol{b} \boldsymbol{c}
$$

但并乘不满足交换律。

后两个式子表示两个矢量的并积是二重线性的。

矢量 $\boldsymbol{a},\boldsymbol{b}$ 的并积也可以通过基矢量的并积表示出来:

$$
\boldsymbol{a} \boldsymbol{b}
= (a^i \boldsymbol{e}_i) (b^j \boldsymbol{e}_j)
= a^i b^j \boldsymbol{e}_i \boldsymbol{e}_j
$$

在三维空间中,两个基矢量的并积 $\boldsymbol{e}_i \boldsymbol{e}_j$ 共有9个,利用它们的线性组合即可定义二阶张量,而并积 $\boldsymbol{a} \boldsymbol{b}$ 是其特例。

\subsection{张量的定义}

如果一个量可以分解为基矢量或其并积的线性组合,并且相应系数在坐标变换时按照逆变公式变换,则这样的量称为张量,而这些系数称为该张量的逆变分量。对于二阶张量,

$$
\boldsymbol{T}
= T^{ij} \boldsymbol{e}_i \boldsymbol{e}_j
= T'^{ij} \boldsymbol{e}'_i \boldsymbol{e}'_j
$$

并且逆变分量 $T^{ij}$在坐标变换时的变换公式为:

$$
T'^{ij}
= \frac{\partial x'^i}{\partial x^k} \frac{\partial x'^j}{\partial x^l} T^{kl}
$$

类似地,对于三阶张量,

$$
\boldsymbol{T}
= T^{ijk} \boldsymbol{e}_i \boldsymbol{e}_j \boldsymbol{e}_k
= T'^{ijk} \boldsymbol{e}'_i \boldsymbol{e}'_j \boldsymbol{e}'_k
$$

并且

$$
T'^{ijk}
= \frac{\partial x'^i}{\partial x^l} \frac{\partial x'^j}{\partial x^m} \frac{\partial x'^k}{\partial x^n}T^{lmn}
$$

总之, $n$ 阶张量是 $n$ 个基矢量的并积的线性组合,并且在坐标变换时保持不变。既然并积是多重线性的, $n$ 阶张量也是多重线性的。一阶张量就是前面定义的矢量,而零阶张量称为标量。

从张量的定义可以看出,尽管张量可以按照坐标系的基矢量或其并积分解,即张量的分量与坐标系有关,张量本身却与坐标系的选择无关。因此,张量是坐标变换下的不变量。用物理学的话来说,物理学定律在不同惯性参考系下应当具有相同的数学表达形式。这种数学形式的不变性其实就是张量在坐标变换下的不变性。而物理学定律的这种性质称为协变性(不要与协变的含义相混淆)。

二阶逆变公式也可以写为矩阵形式,设二阶张量在新、旧坐标系下的逆变分量分别组成矩阵 $\boldsymbol{T}' = (T'^{ij})$ , $\boldsymbol{T} = (T^{ij})$。则:

$$
\boldsymbol{T}' = \boldsymbol{J} \boldsymbol{T} \boldsymbol{J}^T
$$

作为对比,两个基矢量的并积的协变公式

$$
\boldsymbol{e}'_i \boldsymbol{e}'_j
= \frac{\partial x^k}{\partial x'^i} \frac{\partial x^l}{\partial x'^j} \boldsymbol{e}_k \boldsymbol{e}_l
$$

也可以写为矩阵形式:

$$
\boldsymbol{E}' = \boldsymbol{J}^{-1T} \boldsymbol{E} \boldsymbol{J}^{-1}
$$

其中 $\boldsymbol{E}' = (\boldsymbol{e}'_i \boldsymbol{e}'_j)$ , $\boldsymbol{E} = (\boldsymbol{e}_i \boldsymbol{e}_j)$

由此可见协变公式与逆变公式是互逆的。

\subsection{度量张量}

记 $g_{ij} = \boldsymbol{e}_i \cdot \boldsymbol{e}_j$ ,则 $g_{ij}$ 显然是协变的和对称的。由上面的结果,若矩阵 $(g_{ij})$ 的逆矩阵记为 $(g^{ij})$ ,则 $g^{ij}$ 是逆变的。且有:

$$
g^{ij} g_{jk} = \delta^i_k
$$

\begin{remark}
    由上式可推出: $dg_{ij} = -g_{ik} g_{jl} dg^{kl}$
\end{remark}

并且 $g^{ij}$ 定义了一个二阶张量

$$
\boldsymbol{g}
= g^{ij} \boldsymbol{e}_i \boldsymbol{e}_j
$$

称之为度量张量,其为表征空间中的长度与角度的二阶张量,具体公式如下:

$$
a
= |\boldsymbol{a}|
= \sqrt{\boldsymbol{a} \cdot \boldsymbol{a}}
= \sqrt{a^i a^j g_{ij}}
$$

$$
\alpha
= \arccos \frac{\boldsymbol{a} \cdot \boldsymbol{b}}{|\boldsymbol{a}| |\boldsymbol{b}|}
= \arccos \frac{a^i b^j g_{ij}}{\sqrt{a^k a^l g_{kl} b^m b^n g_{mn}}}
$$

\subsection{逆变基矢量 $\;$ 张量的协变、混合分量}

定义逆变基矢量

$$
\boldsymbol{e}^i = g^{ij} \boldsymbol{e}_j
$$

易知

$$
\boldsymbol{e}_i = g_{ij} \boldsymbol{e}^j
$$

$$
\boldsymbol{e}_i \cdot \boldsymbol{e}^j = \delta_i^j
$$

利用逆变基矢量,可以把张量按照逆变基矢量或协变、逆变基矢量的并积分解,相应系数分别称为张量的协变分量和混合分量。

\begin{example}
    对于矢量 $\boldsymbol{A} = A^i \boldsymbol{e}_i$

    $$
    \boldsymbol{A}
    = A^i \boldsymbol{e}_i
    = A_i \boldsymbol{e}^i
    $$

    且逆变、协变分量满足:

    $$
    A_i = g_{ij} A^j \quad A^i = g^{ij} A_j
    $$
\end{example}

\begin{example}
    对于二阶张量 $\boldsymbol{T} = T^{ij} \boldsymbol{e}_i \boldsymbol{e}_j$

    $$
    \boldsymbol{T}
    = T^{ij} \boldsymbol{e}_i \boldsymbol{e}_j
    = T_{ij} \boldsymbol{e}^i \boldsymbol{e}^j
    = T^{i \cdot}_{\cdot j} \boldsymbol{e}_i \boldsymbol{e}^j
    = T^{\cdot j}_{i \cdot} \boldsymbol{e}^i \boldsymbol{e}_j
    $$

    且

    $$
    T_{ij} = g_{ik} g_{jl} T^{kl} \quad
    T^{i \cdot}_{\cdot j} = g_{jk} T^{ik} \quad
    T^{\cdot j}_{i \cdot} = g_{ik} T^{kj}
    $$

    显然这些分量在坐标变换时的变换公式为:

    $$
    T'_{ij}
    = \frac{\partial x^k}{\partial x'^i}  \frac{\partial x^l}{\partial x'^j} T_{kl} \quad
    T'^{i \cdot}_{\cdot j}
    = \frac{\partial x'^i}{\partial x^k}  \frac{\partial x^l}{\partial x'^j} T^{k \cdot}_{\cdot l}\quad
    T'^{\cdot j}_{i \cdot}
    = \frac{\partial x^k}{\partial x'^i} \frac{\partial x'^j}{\partial x^l} T^{\cdot l}_{k \cdot}
    $$
\end{example}

\begin{example}
    度量张量的逆变分量为 $g^{ij}$ ,协变分量为 $g_{ij}$ ,两个混合分量在任意坐标系下均为 $\delta^i_j$,这意味着:

    $$
    \boldsymbol{g}
    = \boldsymbol{e}_i \boldsymbol{e}^i
    = \boldsymbol{e}^i \boldsymbol{e}_i
    $$
\end{example}

利用度量张量的协变分量和逆变分量可以把基矢量和张量分量的角标由上标写为下标或由下标写为上标。对角标的这种运算分别称为降标和升标。

\begin{remark}
    在正交坐标系下还经常把张量按照单位基矢量的并积分解,相应系数称为物理分量。在表示物理分量时,经常把表示坐标的字母直接写为下标。例如在柱面坐标系 $r, \theta, z$ 下,速度 $v$ 的物理分量通常记为 $v_r, v_\theta, v_z$。
    
    使用物理分量的一个好处是,一个量的各物理分量与该量本身具有相同的量纲。但是用物理分量写出的公式通常很繁琐。并且物理分量(以及单位基矢量)通常既不是协变的,也不是逆变的。只有在直角坐标系下,各种分量才没有区别。
\end{remark}

由于坐标 $x^i$ 的指标在上方,部分人会称这样的坐标为“逆变坐标”(请注意,这个说法有问题)。那么对于每个曲线坐标系,是否存在着一组协变坐标 $x_i$ ,满足 $\boldsymbol{e}^i = \frac{\boldsymbol{\partial r}}{\partial x_i}$ ?

如果这样的曲线坐标存在,显然 $x_i$ 与 $x^i$ 之间具有可逆转换关系。于是可得:

$$
\boldsymbol{e}_i
= \frac{\partial \boldsymbol{r}}{\partial x^i}
= \frac{\partial x_j}{\partial x^i} \frac{\partial \boldsymbol{r}}{\partial x_j}
= \frac{\partial x_j}{\partial x^i} \boldsymbol{e}^j
$$
 
上式等价于

$$
\frac{\partial x_j}{\partial x^i} = g_{ij}
$$ 

那么下式须成立:

$$
\frac{\partial g_{ij}}{\partial x^k}
= \frac{\partial}{\partial x^k} (\frac{\partial x_j}{\partial x^i})
= \frac{\partial}{\partial x^i} (\frac{\partial x_j}{\partial x^k})
= \frac{\partial g_{jk}}{\partial x^i}
$$

这个条件并不总成立,例如柱坐标的度量张量分量就不满足上述条件。因此,并不存在所谓的“协变坐标”,故“逆变坐标”的说法同样不妥当。

\section{张量的代数运算}

\subsection{转置}

交换张量分解式中分量的两个角标(或相应的两个基矢量)的位置的运算称为相对于这两个角标的转置,所得张量称为转置张量

\begin{example}
    $\boldsymbol{P} = P^{ij} \boldsymbol{e}_i \boldsymbol{e}_j$ 的转置张量为 $\boldsymbol{P}^T = P^{ji} \boldsymbol{e}_i \boldsymbol{e}_j = P^{ij} \boldsymbol{e}_j \boldsymbol{e}_i$
\end{example}

若转置张量与原张量相等,则该张量称为相对于这两个角标的对称张量。若转置张量与原张量仅仅符号不同,则该张量称为相对于这两个角标的反对称张量。张量的对称与反对称性显然与坐标系无关。

任何张量均可分解为一个对称张量和一个反对称张量的和。比如:

$$
\boldsymbol{P}
= \frac{\boldsymbol{P} + \boldsymbol{P}^T}{2} + \frac{\boldsymbol{P} - \boldsymbol{P}^T}{2}
$$

对二阶张量,这样的分解式是唯一的。

对于张量 $\boldsymbol{P}$ 。运算 $\frac{\boldsymbol{P} + \boldsymbol{P}^T}{2}$ 和 $\frac{\boldsymbol{P} - \boldsymbol{P}^T}{2}$ 分别称为对称化和反对称化。

\subsection{并乘与并积}

若干张量并排写在一起的运算称为并乘,其结果称为并积。

\begin{example}
    $\boldsymbol{P} = P^{ij} \boldsymbol{e}_i \boldsymbol{e}_j$ 与 $\boldsymbol{Q} = Q^{ij} \boldsymbol{e}_i \boldsymbol{e}_j$ 的并积为

    $$
    \boldsymbol{P} \boldsymbol{Q}
    = P^{ij} Q^{kl} \boldsymbol{e}_i \boldsymbol{e}_j \boldsymbol{e}_k \boldsymbol{e}_l
    $$
\end{example}

\subsection{缩并}

让一个张量的混合分量的某个上标与某个下标相等,从而成为表示求和的一对哑标,所得结果是比原张量低二阶的张量的分量,这样的运算称为对这两个角标的缩并,而由此得到的低二阶的张量也称为原张量对这两个角标的缩并。

\begin{example}
    $T^{i j \cdot \cdot}_{\cdot \cdot k l} \boldsymbol{e}_i \boldsymbol{e}_j \boldsymbol{e}_k \boldsymbol{e}_l$ 对第二、第三个角标的缩并是二阶张量 $T^{i j \cdot \cdot}_{\cdot \cdot j l} \boldsymbol{e}_i \boldsymbol{e}_l$ 
\end{example}

缩并运算有时也用于非张量对象,例如包含克氏符号的表达式。

\subsection{点乘与点积}

以二阶张量为例,两个张量的点乘运算(结果为点积)按照以下方式定义:

$$
\boldsymbol{P} \cdot \boldsymbol{Q}
= P^{ij} Q_{kl} \boldsymbol{e}_i \boldsymbol{e}_j \cdot \boldsymbol{e}^k \boldsymbol{e}^l
= P^{ij} Q_{jl} \boldsymbol{e}_i \boldsymbol{e}^l
$$

因此,点乘相当于先并乘,再对第一个张量分量最后一个角标和第二个张量分量的第一个角标进行缩并。特别地,两个矢量的点积是标量。

矢量与基矢量的点积是相应分量:

$$
\boldsymbol{A} \cdot \boldsymbol{e}_i = A_i 
\quad
\boldsymbol{A} \cdot \boldsymbol{e}^i = A^i
$$

任意张量与度量张量的点积还是此张量本身,即度量张量相当于点乘运算的单位因子:

$$
\boldsymbol{T} \cdot \boldsymbol{g}
= \boldsymbol{g} \cdot \boldsymbol{T}
= \boldsymbol{T}
$$

此外,非退化二阶张量 $\boldsymbol{T}$ 的逆张量 $\boldsymbol{T}^{-1}$ 是满足以下关系的张量:

$$
\boldsymbol{T} \cdot \boldsymbol{T}^{-1}
= \boldsymbol{T}^{-1} \cdot \boldsymbol{T}
= \boldsymbol{g}
$$

\begin{remark}
    点乘运算可用以表述张量识别定理(商定理):
    
    如果任意的 $n$ 阶张量 $\boldsymbol{T}$ 与量 $\boldsymbol{X}$ 的点积 $\boldsymbol{T} \cdot \boldsymbol{X}$ 都是 $n + m - 2$ 阶张量,则 $\boldsymbol{X}$ 是 $m$ 阶张量
\end{remark}

有时还会用到双点乘运算,其结果为双点积,定义如下(以二阶张量为例):

$$
\boldsymbol{P} \cdotp \cdot \boldsymbol{Q}
= P^{ij} Q_{ji} \quad
\boldsymbol{P} : \boldsymbol{Q}
= P^{ij} Q_{ij}
$$

\subsection{数值不变量}

若以张量分量为自变量的某个函数的值与坐标系无关,则该函数称为相关张量的数值不变量,简称不变量。

\begin{example}
    矢量 $\boldsymbol{A}$ 的长度 $A = |\boldsymbol{A}| = \sqrt{A^i A_i}$
\end{example}

\begin{example}
    二阶张量的主不变量

    对任意一个二阶张量 $\boldsymbol{T}$ ,记 $I_1, I_2, I_3$ 分别为其第一、第二、第三主不变量,其具体表达式如下:

    \begin{itemize}
        \item $I_1 = T^{i \cdot}_{\cdot i} = T^{\cdot i}_{i \cdot}$
        \item $I_2 = \frac{1}{2} (T^{i \cdot}_{\cdot i} T^{j \cdot}_{\cdot j} - T^{i \cdot}_{\cdot j} T^{j \cdot}_{\cdot i})$
        \item $I_3 = \det(T^{i \cdot}_{\cdot j}) = \det(T^{\cdot j}_{i \cdot})$
    \end{itemize}
\end{example}

\begin{proof}
    对 $I_3$ 证明如下:

    $$
    T'^{i \cdot}_{\cdot j}
    = \frac{\partial x'^i}{\partial x^k} \frac{\partial x^l}{\partial x'^j} T^{k \cdot}_{\cdot l}
    $$

    故:

    $$
    (T'^{i \cdot}_{\cdot j})
    = J \, (T^{i \cdot}_{\cdot j}) \, J^{-1} 
    $$

    得到:

    $$
    \det(T'^{i \cdot}_{\cdot j})
    = \det(T^{i \cdot}_{\cdot j})
    $$

    同理可得

    $$
    \det(T'^{\cdot j}_{i \cdot})
    = \det(T^{\cdot j}_{i \cdot})
    $$

    再由

    $$
    T^{i \cdot}_{\cdot j}
    = g^{ik} g_{jl} T^{\cdot l}_{k \cdot}
    $$

    得到:

    $$
    (T^{i \cdot}_{\cdot j})
    = G^{-1} \, (T^{\cdot j}_{i \cdot}) \, G 
    $$

    故:

    $$
    \det(T^{i \cdot}_{\cdot j})
    = \det(T^{\cdot j}_{i \cdot})
    $$
\end{proof}

由上可以定义二阶张量的迹和行列式:

$$
tr \boldsymbol{T}
= T^{i \cdot}_{\cdot i} = T^{\cdot i}_{i \cdot}
$$

$$
\det \boldsymbol{T}
= \det(T^{i \cdot}_{\cdot j})
= \det(T^{\cdot j}_{i \cdot})
$$

\subsection{矢量的叉积与混合积}

两个矢量 $\boldsymbol{a}, \boldsymbol{b}$ 经过叉乘运算后给出矢量 $\boldsymbol{c} = \boldsymbol{a} \times \boldsymbol{b}$ ,其大小等于以 $\boldsymbol{a}, \boldsymbol{b}$ 为边的平行四边形的面积,其方向垂直于 $\boldsymbol{a}, \boldsymbol{b}$ 所在平面,并且从 $\boldsymbol{c}$ 所指方向看,在该平行四边形内从 $\boldsymbol{a}$ 向 $\boldsymbol{b}$ 环绕的方向为逆时针方向(当 $\boldsymbol{a}$ 与 $\boldsymbol{b}$ 指向相同或相反方向时 $\boldsymbol{c} = 0$ )。

两个矢量叉乘的结果(称为叉积)是唯一确定的,与表示矢量所用的坐标系无关。叉积相对于相乘的矢量是反对称的,即:

$$
\boldsymbol{a} \times \boldsymbol{b}
= - \boldsymbol{b} \times \boldsymbol{a}
$$

两个矢量的叉积与第三个矢量的点积称为这三个矢量的混合积,其绝对值等于以这三个矢量为边的平行六面体的体积。混合积相对于相乘的任何两个矢量都是反对称的。由此可知,相乘的三个矢量在轮换后,其混合积不变,即:

$$
(\boldsymbol{a} \times \boldsymbol{b}) \cdot \boldsymbol{c}
= (\boldsymbol{b} \times \boldsymbol{c}) \cdot \boldsymbol{a}
= (\boldsymbol{c} \times \boldsymbol{a}) \cdot \boldsymbol{b}
$$

对于基矢量,可以类似地定义点积和混合积。前者是度量张量的分量,而后者是置换张量的分量(见下文)。利用置换张量很容易进行与又乘有关的运算。

此外,有以下公式(这些公式在上标变为下标、下标变为上标后也成立):

$$
\boldsymbol{e}^1
= \frac{\boldsymbol{e}_2 \times \boldsymbol{e}_3}{(\boldsymbol{e}_1 \times \boldsymbol{e}_2) \cdot \boldsymbol{e}_3} \quad
\boldsymbol{e}^2
= \frac{\boldsymbol{e}_3 \times \boldsymbol{e}_1}{(\boldsymbol{e}_1 \times \boldsymbol{e}_2) \cdot \boldsymbol{e}_3} \quad
\boldsymbol{e}^3
= \frac{\boldsymbol{e}_1 \times \boldsymbol{e}_2}{(\boldsymbol{e}_1 \times \boldsymbol{e}_2) \cdot \boldsymbol{e}_3}
$$

\begin{proof}
    设 $i, j, k$ 是 $1, 2, 3$ 的排列。根据矢量叉乘的定义, $\boldsymbol{e}_i \times \boldsymbol{e}_j$ 应该与 $\boldsymbol{e}_i$ 和 $\boldsymbol{e}_j$ 都正交,故其与 $\boldsymbol{e}^k$ 共线。设 $\boldsymbol{e}_i \times \boldsymbol{e}_j = \lambda \boldsymbol{e}^k$。对该表达式两侧点乘 $\boldsymbol{e}_k$ (这里取消求和符号)即可得 $\lambda = (\boldsymbol{e}_i \times \boldsymbol{e}_j) \cdot \boldsymbol{e}_k$
\end{proof}

\subsection{置换张量 $\;$ 张量的叉乘与叉积}

定义一个在张量分析中广泛应用的三阶反对称张量———置换张量 $\boldsymbol{\varepsilon}$ ,其协变分量为:

$$
\varepsilon_{ijk}
= (\boldsymbol{e}_i \times \boldsymbol{e}_j) \cdot \boldsymbol{e}_k
$$

逆变分量为:

$$
\varepsilon^{ijk}
= (\boldsymbol{e}^i \times \boldsymbol{e}^j) \cdot \boldsymbol{e}^k
$$

矢量的叉积和混合积均可用置换张量表示出来:

$$
\boldsymbol{e}_i \times \boldsymbol{e}_j
= \varepsilon_{ijk} \boldsymbol{e}^k
$$

$$
\boldsymbol{e}^i \times \boldsymbol{e}^j
= \varepsilon^{ijk} \boldsymbol{e}_k
$$

$$
\boldsymbol{a} \times \boldsymbol{b}
= \varepsilon_{ijk} a^i b^j \boldsymbol{e}^k
= \varepsilon^{ijk} a_i b_j \boldsymbol{e}_k
$$

$$
(\boldsymbol{a} \times \boldsymbol{b}) \cdot \boldsymbol{c}
= \varepsilon_{ijk} a^i b^j c^k
= \varepsilon^{ijk} a_i b_j c_k
$$

特别地,在右手直角坐标系下有:

$$
\det(T^{i \cdot}_{\cdot j})
= \varepsilon_{ijk} T^{i \cdot}_{\cdot 1} T^{j \cdot}_{\cdot 2}T^{k \cdot}_{\cdot 3}
= \varepsilon^{ijk} T^{1 \cdot}_{\cdot i} T^{2 \cdot}_{\cdot j}T^{3 \cdot}_{\cdot k}
$$

故在右手直角坐标系下可以用行列式的形式表示矢量的叉积和混合积:

$$
\boldsymbol{a} \times \boldsymbol{b}
= \begin{vmatrix}
    \boldsymbol{i} & \boldsymbol{j} & \boldsymbol{k} \\
    a_1 & a_2 & a_3 \\
    b_1 & b_2 & b_3
\end{vmatrix}
$$

$$
(\boldsymbol{a} \times \boldsymbol{b}) \cdot \boldsymbol{c}
= \begin{vmatrix}
    a_1 & a_2 & a_3 \\
    b_1 & b_2 & b_3 \\
    c_1 & c_2 & c_3
\end{vmatrix}
$$

我们利用上述结果给出置换张量与度量张量的关系,取右手直角坐标系 $x, y, z$ ,那么:

$$
(\boldsymbol{e}_1 \times \boldsymbol{e}_2) \cdot \boldsymbol{e}_3
= \begin{vmatrix}
    e_{1x} & e_{1y} & e_{1z} \\
    e_{2x} & e_{2y} & e_{2z} \\
    e_{3x} & e_{3y} & e_{3z}
\end{vmatrix}
$$

$$
[(\boldsymbol{e}_1 \times \boldsymbol{e}_2) \cdot \boldsymbol{e}_3]^2
= \begin{vmatrix}
    e_{1x} & e_{1y} & e_{1z} \\
    e_{2x} & e_{2y} & e_{2z} \\
    e_{3x} & e_{3y} & e_{3z}
\end{vmatrix}
\begin{vmatrix}
    e_{1x} & e_{2x} & e_{3x} \\
    e_{1y} & e_{2y} & e_{3y} \\
    e_{1z} & e_{2z} & e_{3z}
\end{vmatrix}
= \begin{vmatrix}
    \boldsymbol{e}_1 \cdot \boldsymbol{e}_1 & \boldsymbol{e}_1 \cdot \boldsymbol{e}_2 & \boldsymbol{e}_1 \cdot \boldsymbol{e}_3 \\
    \boldsymbol{e}_2 \cdot \boldsymbol{e}_1 & \boldsymbol{e}_2 \cdot \boldsymbol{e}_2 & \boldsymbol{e}_2 \cdot \boldsymbol{e}_3 \\
    \boldsymbol{e}_3 \cdot \boldsymbol{e}_1 & \boldsymbol{e}_3 \cdot \boldsymbol{e}_2 & \boldsymbol{e}_3 \cdot \boldsymbol{e}_3
\end{vmatrix}
= \det(g_{ij})
$$

记 $g = \det(g_{ij})$ ,则:

$$
\varepsilon_{ijk}
= (\boldsymbol{e}_i \times \boldsymbol{e}_j) \cdot \boldsymbol{e}_k
= \begin{cases}
    \sqrt{g} & \boldsymbol{e}_i, \boldsymbol{e}_j, \boldsymbol{e}_k \text{组成右手系} \\
    -\sqrt{g} & \boldsymbol{e}_i, \boldsymbol{e}_j, \boldsymbol{e}_k \text{组成左手系} \\
    0 & i,j,k \text{中有相同的数}
\end{cases}
$$

类似地:

$$
\varepsilon^{ijk}
= (\boldsymbol{e}^i \times \boldsymbol{e}^j) \cdot \boldsymbol{e}^k
= \begin{cases}
    \frac{1}{\sqrt{g}} & \boldsymbol{e}^i, \boldsymbol{e}^j, \boldsymbol{e}^k \text{组成右手系} \\
    - \frac{1}{\sqrt{g}} & \boldsymbol{e}^i, \boldsymbol{e}^j, \boldsymbol{e}^k \text{组成左手系} \\
    0 & i,j,k \text{中有相同的数}
\end{cases}
$$

由此可以用置换张量定义二阶张量行列式的一般表达式,此式在任意坐标系下都成立:

$$
\det \boldsymbol{T}
= \det(T^{i \cdot}_{\cdot j})
= \det(T^{\cdot j}_{i \cdot})
= \varepsilon_{ijk} \varepsilon^{123} T^{i \cdot}_{\cdot 1} T^{j \cdot}_{\cdot 2}T^{k \cdot}_{\cdot 3}
= \frac{1}{6} \varepsilon_{ijk} \varepsilon^{\alpha \beta \gamma} T^{i \cdot}_{\cdot \alpha} T^{j \cdot}_{\cdot \beta}T^{k \cdot}_{\cdot \gamma}
$$

\begin{proof}
    设置换符号

    $$
    \epsilon_{i_1 i_2 \dots i_n} =
    \begin{cases}
        0 & \text{存在相同的} i_k \\
        1 & i_1 i_2 \dots i_n \text{是偶排列} \\
        -1 & i_1 i_2 \dots i_n \text{是奇排列}
    \end{cases}
    $$

    则:

    $$
    \det \boldsymbol{T}
    = \epsilon_{ijk} T^{i \cdot}_{\cdot 1} T^{j \cdot}_{\cdot 2}T^{k \cdot}_{\cdot 3}
    = \pm \frac{1}{\sqrt{g}} \varepsilon_{ijk} T^{i \cdot}_{\cdot 1} T^{j \cdot}_{\cdot 2}T^{k \cdot}_{\cdot 3}
    = \varepsilon_{ijk} \varepsilon^{123} T^{i \cdot}_{\cdot 1} T^{j \cdot}_{\cdot 2}T^{k \cdot}_{\cdot 3}
    $$

    考虑到 $\alpha, \beta, \gamma$ 是 $1, 2, 3$ 的排列时, $\varepsilon^{\alpha \beta \gamma} T^{i \cdot}_{\cdot \alpha} T^{j \cdot}_{\cdot \beta}T^{k \cdot}_{\cdot \gamma} = \varepsilon^{123} T^{i \cdot}_{\cdot 1} T^{j \cdot}_{\cdot 2}T^{k \cdot}_{\cdot 3}$ 。故:

    $$
    \det \boldsymbol{T}
    = \frac{1}{6} \varepsilon_{ijk} \varepsilon^{\alpha \beta \gamma} T^{i \cdot}_{\cdot \alpha} T^{j \cdot}_{\cdot \beta}T^{k \cdot}_{\cdot \gamma}
    $$
\end{proof}

两个张量的叉乘运算可以通过这两个张量与置换张量的点乘运算来定义:

$$
\boldsymbol{P} \times \boldsymbol{Q}
= - \boldsymbol{P} \cdotp \boldsymbol{\varepsilon} \cdotp \boldsymbol{Q}
$$

\begin{example}
    对两个二阶张量:

    $$
    \boldsymbol{P} \times \boldsymbol{Q}
    = P^{ij} Q^{kl} \boldsymbol{e}_i \boldsymbol{e}_j \times \boldsymbol{e}_k \boldsymbol{e}_l
    = P^{ij} Q^{kl} \varepsilon_{jkm} \boldsymbol{e}_i \boldsymbol{e}^m \boldsymbol{e}_l
    = - \boldsymbol{P} \cdotp \boldsymbol{\varepsilon} \cdotp \boldsymbol{Q}
    $$
\end{example}

可以证明,置换张量的分量满足以下恒等式:

$$
\varepsilon^{ijk} \varepsilon_{pqr}
= \begin{vmatrix}
    \delta^i_p & \delta^i_q & \delta^i_r \\
    \delta^j_p & \delta^j_q & \delta^j_r \\
    \delta^k_p & \delta^k_q & \delta^k_r
\end{vmatrix}
$$

$$
\varepsilon^{ijk} \varepsilon_{pqk}
= \begin{vmatrix}
    \delta^i_p & \delta^i_q \\
    \delta^j_p & \delta^j_q
\end{vmatrix}
$$

$$
\varepsilon^{ijk} \varepsilon_{pjk}
= 2 \delta^i_p
$$

$$
\varepsilon^{ijk} \varepsilon_{ijk}
= 6
$$

\begin{remark}
    方法:将各置换张量分量表示为由基矢量在直角坐标系的分量组成的行列式
\end{remark}

下面给出上述恒等式的一个应用:

\begin{example}
    设矢量 $\boldsymbol{a}, \boldsymbol{b}, \boldsymbol{c}$ ,求证 $\boldsymbol{a} \times (\boldsymbol{b} \times \boldsymbol{c}) = \boldsymbol{a} \cdot \boldsymbol{c}\boldsymbol{b} - \boldsymbol{a} \cdot \boldsymbol{b}\boldsymbol{c}$

    证明:

    $$
    \boldsymbol{a} \times (\boldsymbol{b} \times \boldsymbol{c})
    = a^p \boldsymbol{e}_p \times \varepsilon^{ijk} b_i c_j \boldsymbol{e}_k
    = \varepsilon_{rpq} \varepsilon^{ijq} a^p b_i c_j \boldsymbol{e}^r
    $$

    $$
    \Rightarrow
    \boldsymbol{a} \times (\boldsymbol{b} \times \boldsymbol{c})
    = (\delta^i_r \delta^j_p - \delta^i_p \delta^j_r) a^p b_i c_j \boldsymbol{e}^r
    = a^j b_i c_j \boldsymbol{e}^i - a^i b_i c_j \boldsymbol{e}^j
    $$

    $$
    \Rightarrow
    \boldsymbol{a} \times (\boldsymbol{b} \times \boldsymbol{c})
    = \boldsymbol{a} \cdot \boldsymbol{c}\boldsymbol{b} - \boldsymbol{a} \cdot \boldsymbol{b}\boldsymbol{c}
    $$
\end{example}

\section{张量的微分运算}

\subsection{第二类克氏符号}

实际运算中往往会遇到基矢量对坐标的偏导数,比如在速度的物质导数公式中,有:

$$
\frac{\partial \boldsymbol{v}}{\partial x^i}
= \frac{\partial (v^k \boldsymbol{e}_k)}{\partial x^i}
= \frac{\partial v^k}{\partial x^i} \boldsymbol{e}_k + v^k \frac{\partial \boldsymbol{e}_k}{\partial x^i}
$$

协变基矢量对坐标的偏导数可以按照各协变基矢量分解,通常用记号 $\Gamma^j_{ki}$ 表示相应的系数:

$$
\frac{\partial \boldsymbol{e}_j}{\partial x^i}
= \Gamma^j_{ki} \boldsymbol{e}_k
$$

这些系数称为第二类克里斯托费尔符号(第二类克氏符号)

\begin{remark}
    第一类克氏符号: $\Gamma_{kij} = \Gamma^m_{ij} g_{mk}$
\end{remark}

\begin{example}
    对柱面坐标系 $r,\theta,z$:

    $$
    \Gamma^1_{22} = -r \quad
    \Gamma^2_{21} = \Gamma^2_{12} = \frac{1}{r} \quad
    \text{其余为零}
    $$
\end{example}

由上式可以推出逆变基矢量对坐标的偏导数:

\begin{gather*}
    \delta^k_j = \boldsymbol{e}_j \cdot \boldsymbol{e}^k \\
    \Rightarrow
    0 = \frac{\partial}{\partial x^i} (\boldsymbol{e}_j \cdot \boldsymbol{e}^k)
    = \boldsymbol{e}_j \cdot \frac{\partial \boldsymbol{e}^k}{\partial x^i} + \frac{\partial \boldsymbol{e}_j}{\partial x^i} \cdot \boldsymbol{e}^k
    = \boldsymbol{e}_j \cdot \frac{\partial \boldsymbol{e}^k}{\partial x^i} + \Gamma^k_{ji}
\end{gather*}

$$
\Rightarrow
\frac{\partial \boldsymbol{e}^k}{\partial x^i}
= - \Gamma^k_{ji} \boldsymbol{e}^j
$$

可以证明, $\Gamma^k_{ij}$ 不是任何张量的分量。

我们知道, $\boldsymbol{e}_i = \frac{\partial \boldsymbol{r}}{\partial x^i}$ ,故有:

$$
\frac{\partial \boldsymbol{e}_k}{\partial x^i}
= \frac{\partial \boldsymbol{e}_i}{\partial x^k}
(= \frac{\partial^2 \boldsymbol{r}}{\partial x^i \partial x^k})
$$

进而可得:

$$
\Gamma^k_{ij} = \Gamma^k_{ji}
$$

即 $\Gamma^k_{ij}$ 关于两个下标是对称的

此外,第二类克氏符号与度量张量的分量有如下关系:

$$
\Gamma^k_{ij}
= \frac{1}{2} g^{k \alpha} (\frac{\partial g_{i \alpha}}{\partial x^j} + \frac{\partial g_{j \alpha}}{\partial x^i} - \frac{\partial g_{i j}}{\partial x^\alpha})
$$

\begin{proof}
    $$
\Gamma^k_{ij}
= \boldsymbol{e}^k \cdot \frac{\partial \boldsymbol{e}_i}{\partial x^j}
= g^{k \alpha} \boldsymbol{e}_\alpha \cdot \frac{\partial \boldsymbol{e}_i}{\partial x^j}
= g^{k \alpha} (\frac{\partial g_{i \alpha}}{\partial x^j} - \boldsymbol{e}_i \cdot \frac{\partial \boldsymbol{e}_\alpha}{\partial x^j})
    $$

    $$
\Rightarrow
\Gamma^k_{ij}
= g^{k \alpha} (\frac{\partial g_{i \alpha}}{\partial x^j} - \boldsymbol{e}_i \cdot \frac{\partial \boldsymbol{e}_j}{\partial x^\alpha})
    $$

    同理:
    
    $$
    \Gamma^k_{ij} = \Gamma^k_{ji}
    = g^{k \alpha} (\frac{\partial g_{j \alpha}}{\partial x^i} - \boldsymbol{e}_j \cdot \frac{\partial \boldsymbol{e}_i}{\partial x^\alpha})
    $$

    $$
    \Rightarrow
    \Gamma^k_{ij}
    = \frac{1}{2} g^{k \alpha} (\frac{\partial g_{i \alpha}}{\partial x^j} + \frac{\partial g_{j \alpha}}{\partial x^i} - \boldsymbol{e}_i \cdot \frac{\partial \boldsymbol{e}_j}{\partial x^\alpha} - \boldsymbol{e}_j \cdot \frac{\partial \boldsymbol{e}_i}{\partial x^\alpha})
    $$

    $$
    \Rightarrow
    \Gamma^k_{ij}
    = \frac{1}{2} g^{k \alpha} (\frac{\partial g_{i \alpha}}{\partial x^j} + \frac{\partial g_{j \alpha}}{\partial x^i} - \frac{\partial g_{i j}}{\partial x^\alpha})
    $$
\end{proof}

\subsection{协变微分算子 $\;$ 协变导数}

考虑矢量 $\boldsymbol{v}$ 对坐标的偏导数:

$$
\frac{\partial \boldsymbol{v}}{\partial x^i}
= \frac{\partial v^k}{\partial x^i} \boldsymbol{e}_k + v^k \frac{\partial \boldsymbol{e}_k}{\partial x^i}
= (\frac{\partial v^j}{\partial x^i} + v^k \Gamma^j_{ki}) \boldsymbol{e}_j
$$

记

$$
\nabla_i v^j = \frac{\partial v^j}{\partial x^i} + v^k \Gamma^j_{ki}
$$

则

$$
\frac{\partial \boldsymbol{v}}{\partial x^i}
= \nabla_i v^j \boldsymbol{e}_j
$$

这里的 $\nabla_i$ 其实是一个微分算子。当它作用于带有一个上标的量时,其定义由上式给出。

利用点乘运算容易得到:

$$
\nabla_i v^j = \frac{\partial \boldsymbol{v}}{\partial x^i} \cdot \boldsymbol{e}^j
$$

所以 $\nabla_i v^j$ 对下标 $i$ 是协变的,对上标 $j$ 是逆变的。这个量是一个二阶张量的一种混合分量,而该张量显然就是:

$$
\boldsymbol{e}^i \frac{\partial \boldsymbol{v}}{\partial x^i}
= \nabla_i v^j \boldsymbol{e}^i \boldsymbol{e}_j
$$

因此,算子 $\nabla_i$ 称为对坐标 $x^i$ 的协变微分算子, $\nabla_i v^j$ 称为矢量 $\boldsymbol{v}$ 的逆变分量 $v^j$ 对坐标 $x^i$ 的协变导数。

类似地,考虑 $\boldsymbol{v}$ 的协变分量,有:

$$
\frac{\partial \boldsymbol{v}}{\partial x^i}
= \frac{\partial v_k}{\partial x^i} \boldsymbol{e}^k + v_k \frac{\partial \boldsymbol{e}^k}{\partial x^i}
= (\frac{\partial v_j}{\partial x^i} - v_k \Gamma^k_{ji}) \boldsymbol{e}^j
$$

记

$$
\nabla_i v_j = \frac{\partial v_j}{\partial x^i} - v_k \Gamma^k_{ji}
$$

则

$$
\frac{\partial \boldsymbol{v}}{\partial x^i}
= \nabla_i v_j \boldsymbol{e}^j
$$

且

$$
\nabla_i v_j = \frac{\partial \boldsymbol{v}}{\partial x^i} \cdot \boldsymbol{e}_j
$$

于是,当 $\nabla_i$ 作用于带有一个下标的量时,其定义由上式给出。且 $\nabla_i v_j$ 是上述同一个二阶张量的协变分量:

$$
\boldsymbol{e}^i \frac{\partial \boldsymbol{v}}{\partial x^i}
= \nabla_i v_j \boldsymbol{e}^i \boldsymbol{e}^j
$$

$\nabla_i v_j$ 称为矢量 $\boldsymbol{v}$ 的协变分量 $v_j$ 对坐标 $x^i$ 的协变导数。二阶张量 $\boldsymbol{e}^i \frac{\partial \boldsymbol{v}}{\partial x^i}$ 称为矢量 $\boldsymbol{v}$ 的梯度。

对张量的分量可以用类似的方法引入协变导数,以下以二阶张量 $\boldsymbol{T}$ 为例:

$$
\nabla_i T^{jk} = \frac{\partial T^{jk}}{\partial x^i} + T^{\alpha k} \Gamma^j_{\alpha i} + T^{j \alpha} \Gamma^k_{\alpha i}
$$

$$
\nabla_i T_{jk} = \frac{\partial T_{jk}}{\partial x^i} - T_{\alpha k} \Gamma^{\alpha}_{j i} - T_{j \alpha} \Gamma^{\alpha}_{k i}
$$

$$
\nabla_i T^{j \cdot}_{\cdot k} = \frac{\partial T^{j \cdot}_{\cdot k}}{\partial x^i} + T^{\alpha \cdot}_{\cdot k} \Gamma^j_{\alpha i} - T^{j \cdot}_{\cdot \alpha} \Gamma^{\alpha}_{k i}
$$
$$
\nabla_i T^{\cdot k}_{j \cdot} = \frac{\partial T^{\cdot k}_{j \cdot}}{\partial x^i} - T^{\cdot k}_{\alpha \cdot} \Gamma^{\alpha}_{j i} + T^{\cdot \alpha}_{j \cdot} \Gamma^k_{\alpha i}
$$

我们还规定,标量 $\varphi$ 的协变导数就是普通的偏导数:

$$
\nabla_i \varphi
= \frac{\partial \varphi}{\partial x^i} 
$$

可以看出,协变微分算子的具体表达式与作用对象的角标数目 $n$ 有关。其结果为普通的偏导数与 $n$ 个以第二类克氏符号为系数的同类对象(包括自身)的线性组合项的和或差,并且和与差分别对应角标是上标与下标的情况。

容易验证,和与积的协变导数满足下式:

$$
\nabla_i (A^j + B^j) = \nabla_i A^j + \nabla_i  B^j
$$

$$
\nabla_i (A^j * B^{kl}) = \nabla_i A^j * B^{kl} + A^j * \nabla_i B^{kl}
\qquad( * \text{表示并乘、点乘、叉乘等各种乘法运算})
$$

此外,协变、逆变基矢量满足:

$$
\nabla_i \boldsymbol{e}_j = 0 \quad
\nabla_i \boldsymbol{e}^j = 0
$$

由此还可得到:

$$
\nabla_i g^{jk} = 0 \quad
\nabla_i g_{jk} = 0 
$$

$$
\nabla_i \varepsilon^{jkl} = 0 \quad
\nabla_i \varepsilon_{jkl} = 0 
$$

上述诸式表明在进行协变微分运算时,基矢量、度量张量的分量和置换张量的分量可以移入或移出协变微分算子的作用对象而不影响计算结果。

还容易验证,缩并运算与协变微分运算可以交换顺序。例如,为了计算 $\nabla_i T^{j k \cdot}_{\cdot \cdot k}$ ,可以先计算 $\nabla_i T^{j k \cdot}_{\cdot \cdot l}$ 再取缩并,也可以先缩并 $T^{j k \cdot}_{\cdot \cdot l}$ 再计算其协变导数。由此可知,不含自由角标的表达式的协变导数就是普通的偏导数。

\subsection{场论算子}

利用协变微分算子可以定义以下算子:

\begin{table}[htbp]
\centering
\caption{场论算子定义}
\label{tab:field-operators}
\begin{tabular}{cp{7cm}}
\toprule
\textbf{名称} & \textbf{定义} \\
\midrule
逆变微分算子 & $\nabla^i = g^{ik} \nabla_k$ \\
梯度算子 & $\mathrm{grad} = \nabla = \boldsymbol{e}^i \nabla_i$ \\
张量的梯度 & $\mathrm{grad} \, \boldsymbol{T} = \nabla \boldsymbol{T} = \nabla_i T_{jk} \boldsymbol{e}^i \boldsymbol{e}^j \boldsymbol{e}^k$ \\
张量的散度 & $\mathrm{div} \, \boldsymbol{T} = \nabla \cdot \boldsymbol{T} = \nabla_i T^{ij} \boldsymbol{e}_j$ \\
张量的旋度 & $\mathrm{rot} \, \boldsymbol{T} = \mathrm{curl} \, \boldsymbol{T} = \nabla \times \boldsymbol{T} = \varepsilon^{ijk} \nabla_i T_{jl} \boldsymbol{e}_k \boldsymbol{e}^l$ \\
拉普拉斯算子 & $\Delta = \nabla^i \nabla_i$ \\
\bottomrule
\end{tabular}
\end{table}

梯度算子也可以称作哈密顿算子。

\begin{remark}
    一些文献中将上述梯度、散度、旋度称为左梯度、左散度、左旋度,以区别于右梯度、右散度、右旋度。例如矢量的右梯度被定义为:

    $$
    \boldsymbol{v} \nabla
    = \nabla_j v_i \boldsymbol{e}^i \boldsymbol{e}^j
    $$

    这种符号可能引起混淆,因为算子习惯上应写在被作用量的左边。为表示矢量的右梯度,只须使用其左梯度的转置 $(\nabla \boldsymbol{v})^T$ 即可。但这种符号在文献中比较常见,需要读者注意分辨
\end{remark}

标量 $\varphi$ 的梯度 $\nabla \varphi$ 描述 $\varphi$ 在空间中分布的不均匀性。 $\nabla \varphi$ 指向变化最快的方向。张量 $\boldsymbol{T}$ 的梯度 $\nabla \boldsymbol{T}$ 同样描述 $\boldsymbol{T}$ 在空间中分布的不均匀性。当张量 $\boldsymbol{T}$ 在空间中均匀分布时,其梯度为零,所以张量 $\boldsymbol{T}$ 的任何分量的全部协变导数均为零。反之亦然。因此,根据上一节的结果,度量张量和置换张量在空间中都是均匀分布的,它们处处相同。

我们在欧拉观点下计算物质导数时已经遇到梯度算子,因为物质导数公式可以写为:

$$
\frac{d A}{d t}
= \frac{\partial A}{\partial t} + v^i \nabla_i A
= \frac{\partial A}{\partial t} + \boldsymbol{v} \cdot \nabla \boldsymbol{A}
$$

其中 $A$ 可以是标量、矢量或张量。我们在描述变形时还会用到位移、速度等矢量的梯度。

在一些文献中也使用旋度的另一个记号 $\mathrm{curl}$ 。旋度为零的矢量场称为无旋场,旋度不为零的矢量场称为有旋场。速度矢量 $\boldsymbol{v}$ 的旋度称为涡量,通常记为 $\boldsymbol{\omega}$ 。涡量场描述连续介质的局部转动特性(详见后文)。

\begin{example}
    \begin{gather*}
        \nabla \boldsymbol{r}
        = \boldsymbol{e}^i \nabla_i \boldsymbol{r}
        = \boldsymbol{e}^i \boldsymbol{e}_i
        = \boldsymbol{g} \\
        \Rightarrow
        \nabla_i r_j
        = g_{ij} \quad
        \nabla_i r^j
        = \delta_i^j \\
        \nabla \cdot \boldsymbol{r}
        = \nabla_i r^i
        = \delta_i^i
        = 3 \\
        \nabla \times \boldsymbol{r}
        = \varepsilon^{ijk} \nabla_i r_j \boldsymbol{e}_k
        = \varepsilon^{ijk} g_{ij} \boldsymbol{e}_k
        = 0 \\
        \text{(上式利用了置换张量的反对称性与度量张量的对称性)}
    \end{gather*}
\end{example}

\begin{example}
    设 $r$ 是径矢 $\boldsymbol{r}$ 的长度,则:

    $$
    \nabla r = \frac{\boldsymbol{r}}{r} \quad
    \Delta r = \frac{2}{r}
    $$

    $$
    \nabla \frac{1}{r} = - \frac{\boldsymbol{r}}{r^3} \quad
    \Delta \frac{1}{r} = 0
    $$

    由上可知 $\frac{1}{r}$ 是调和函数。
\end{example}

\begin{example}
    速度的物质导数可以写为以下形式:

    $$
    \frac{d \boldsymbol{v}}{dt}
    = \frac{\partial \boldsymbol{v}}{\partial t} + \frac{1}{2} \mathrm{grad} \, \boldsymbol{v}^2 + \boldsymbol{\omega} \times \boldsymbol{v}
    $$
\end{example}

\begin{example}
    $\boldsymbol{r}$ 为径矢, $\boldsymbol{\Omega}$ 为均匀矢量场,则:

    $$
    \mathrm{rot} \, (\boldsymbol{r} \times \boldsymbol{\Omega})
    = -2 \boldsymbol{\Omega}
    $$
\end{example}

\begin{example}
    张量 $\boldsymbol{T}$ 满足公式:

    $$
    \mathrm{rot} \, \mathrm{rot} \, \boldsymbol{T}
    = \mathrm{grad} \, \mathrm{div} \, \boldsymbol{T} - \Delta \boldsymbol{T}
    $$
\end{example}

\subsection{奥高公式}

设 $V$ 是三维欧氏空间中的有界闭区域,其边界 $\Sigma$ 是分片光滑曲面, $\boldsymbol{n}$ 是边界 $\Sigma$ 上的单位外法向关量, $\boldsymbol{T}$ 是区域 $V$ 中的光滑张量场,则奥高公式(Gauss-Ostrogradskii)成立:

$$
\int_V \nabla * \boldsymbol{T} \, dV
= \int_\Sigma \boldsymbol{n} * \boldsymbol{T} \, d \Sigma
$$

上式中的 $*$ 表示并乘、点乘或叉乘

\begin{proof}
    在笛卡尔直角坐标系中考虑式子的左边:

    \begin{align*}
        \int_V \nabla * \boldsymbol{T} \, dV
        &= \int_V \frac{\partial}{\partial x^k} (T_{i_1 i_2 \dots i_n}) \boldsymbol{e}^k * \boldsymbol{e}^{i_1} \boldsymbol{e}^{i_2} \dots \boldsymbol{e}^{i_n} \, dV \\
        &= \left( \int_V \frac{\partial}{\partial x^k} (\delta^k_s T_{i_1 i_2 \dots i_n}) dV \right) \boldsymbol{e}^s * \boldsymbol{e}^{i_1} \boldsymbol{e}^{i_2} \dots \boldsymbol{e}^{i_n} \\
    \end{align*}

    记 $f^k = \delta^k_s T_{i_1 i_2 \dots i_n} \quad \boldsymbol{n} = n_i \boldsymbol{e}^i$ ,由数学分析中的奥高公式:

    $$
    \int_V \frac{\partial f^k}{\partial x^k} \, dV
    = \int_\Sigma n_k f^k \, d \Sigma
    $$

    \begin{align*}
        \Rightarrow
        \int_V \nabla * \boldsymbol{T} \, dV
        &= (\int_V n_k \delta^k_s T_{i_1 i_2 \dots i_n} d \Sigma) \boldsymbol{e}^s * \boldsymbol{e}^{i_1} \boldsymbol{e}^{i_2} \dots \boldsymbol{e}^{i_n} \\
        &= \int_V (n_s \boldsymbol{e}^s) * (T_{i_1 i_2 \dots i_n} \boldsymbol{e}^{i_1} \boldsymbol{e}^{i_2} \dots \boldsymbol{e}^{i_n}) d \Sigma \\
        &= \int_\Sigma \boldsymbol{n} * \boldsymbol{T} \, d \Sigma
    \end{align*}
\end{proof}

取 $*$ 为点乘,则 $\int_V \nabla \cdot \boldsymbol{T} \, dV = \int_\Sigma \boldsymbol{n} \cdot \boldsymbol{T} \, d \Sigma$ ,即张量场的散度在该区域上的积分等于该张量场通过区域边界的通量。这个公式在连续介质力学中大有用处,因为普遍成立的几个守恒定律在应用于物质体或控制体时总是包含上式左边的面积分,把这样的面积分按照高斯公式(在相应条件成立的前提下)转化为体积分对后续分析极其重要。

特别地,当 $\boldsymbol{T}$ 是度量张量 $\boldsymbol{g}$ 时, $\boldsymbol{n} \cdot \boldsymbol{g} = \boldsymbol{n}$ ,而 $\nabla \cdot \boldsymbol{g} = 0$ ,所以:

$$
\int_\Sigma \boldsymbol{n} \, d\Sigma = 0
$$

这个结果将用于引入连续介质力学中最重要的概念——应力张量。

\section{变形的描述}

\subsection{变形的含义}

连续介质的运动在一般情况下远比刚体复杂,因为各物质点之间的距离在运动中能够发生变化,从而导致物质线长度、两条物质线之间夹角、物质面面积和物质体体积的变化以及物质线、物质面和物质体的形状的变化,物质对象的这些变化统称为变形。

一段细皮筋的均匀拉伸是一种最简单的变形。忽略皮筋的粗细,设一段皮筋最初位于坐标轴 $x^1$ 上的线段 $\mathring{A} \mathring{B}$ ,后来位于线段 $AB$ ,它们的长度分别为 $3cm$ 和 $6 cm$ 。不失一般性,设点 $A$ 是坐标轴 $x^1$ 的原点。取皮筋 $\mathring{A} \mathring{B}$ 各点的坐标 $x^1$ 为拉格朗日坐标 $\xi^1$ 。显然可以认为,各物质点在变形前后分别组成一条坐标轴,我们称之为拉格朗日坐标轴。拉格朗日坐标轴最初与坐标轴 $x^1$ 重合(原点和单位都相同),后来随各物质点的移动而一起移动,其原点移动到点 $A$,单位变为原来的 $2$ 倍。由于变形是均匀的,所以每一小段皮筋的伸长因数(长度变化值与原长度之比)都等于 $1$ 。

关于皮筋变形的信息完全包含在变形前后的拉格朗日坐标轴中,拉格朗日坐标轴的变化与皮筋的变形是完全等价的,我们可以通过对比拉格朗日坐标轴基矢量长度的变化来描述变形。

\begin{remark}
    变形与『拉格朗日坐标轴的变化/物质线长度、夹角的变化/拉格朗日坐标系的度量张量分量的变化/ $\xi^i$ 坐标线的长度伸缩、夹角变化』是完全等价的。
\end{remark}

不过,如果原来绷直的细皮筋变成弯曲的,例如紧贴在一个圆柱体的侧面上,情况就复杂了。设皮筋长度不变,构成一段圆弧 $A' B'$ 。此时新的拉格朗日坐标轴是一条曲线坐标轴,但是由于皮筋长度不变,所以这条曲线坐标轴的单位仍然是 $1$ ,即基矢量的长度仍然是 $1$ 。各物质点之间的距离显然变小了,皮筋有变形,但是基矢量的长度没有变化,这是否与上面所说矛盾?

其实,这里并没有矛盾。从几何角度说,一条曲线是一个一维非欧式空间。在这个空间中,两个点之间的距离是这两个点之间的曲线段的长度(所以我们说非欧式空间是弯曲的,而欧氏空间是平坦的)。但是,变形是三维物理空间中的概念,必须在三维欧氏空间中加以考虑,即使考虑圆弧 $A' B'$ 的情况,至少也要在二维欧氏空间(平面)内讨论。这时两个点之间的距离是这两个点之间的直线段的长度,而在更精细的讨论中需要对比的是最初的细长矩形和后来的“扇形”。

按照定义,在欧氏空间中可以引入一个适用于空间中全部点的笛卡儿坐标系直线满足这个要求。所以我们可以利用一维空间中的拉格朗日坐标来描述皮筋沿直线的变形。曲线显然不满足这个要求,所以必须在更高维的空间中才能描述皮筋的弯曲,我们在下文中还会用更深入的张量语言讨论欧氏空间。

\subsection{初始构形与瞬时构形}

推广上面的例子到一般情形,就得到在拉格朗日观点下描述变形一种通用方法:

考虑一团连续介质相对于这团介质的某一个固定状态的变形。为明确起见,考虑初始时刻 $t_0$ 和时刻 $t$ ,并把这团介质分别在这两个时刻所占据的空间区域称为初始构形和瞬时构形,分别记为 $\mathring{K}$ 和 $K$ 。取空间中的曲线坐标系 $x^i$ ,认为各物质点在初始时刻的坐标 $x^i$ 是拉格朗日坐标 $\xi^i$ 。我们的目标是定量地刻画同一个物质对象在这两个时刻的变化,从而描述与变形有关的各种运动学特征。

同一个物质点的拉格朗日坐标在运动过程中始终保持不变,这个性质让我们能够直接利用拉格朗日坐标来描述变形,即在拉格朗日观点下描述变形。具体来说,拉格朗日坐标同样组成一个坐标系,我们称之为拉格朗日坐标系或随体坐标系。根据拉格朗日坐标的上述取法,拉格朗日坐标系 $\xi^i$ 在初始时刻与空间坐标系 $x^i$ 相同。在拉格朗日坐标系中,组成坐标线的物质点始终组成坐标线,但是坐标线本身随着物质点的运动而运动,坐标线的形状和长度都随运动而变化,所以拉格朗日坐标系一般是曲线坐标系。拉格朗日坐标系是在描述刚体运动时广泛使用的固连坐标系的推广,其引入为我们描述变形带来巨大便利。

在拉格朗日观点下,我们沿用初始构形和瞬时构形的术语,但它们指相应的物质体,即初始构形指初始时刻 $t_0$ 的物质体,而瞬时构形指在初始时刻构成初始构形的物质点在时刻 $t$ 所构成的物质体。设初始构形的物质点具有径矢 $\mathring{\boldsymbol{r}}$ ,而初始构形的随体坐标系具有以下属性(根据拉格朗日坐标的定义,这也是曲线坐标系 $x^i$ 的属性):

\begin{itemize}
    \item 协变基矢量 $\mathring{\boldsymbol{e}}_i = \frac{\partial \mathring{\boldsymbol{r}}}{\partial \xi^i}$ ,逆变基矢量 $\mathring{\boldsymbol{e}}^i$
    \item 度量张量的协变分量 $\mathring{g}_{ij}$ 和逆变分量 $\mathring{g}^{ij}$
    \item 置换张量的协变分量 $\mathring{\varepsilon}_{ijk}$ 和逆变分量 $\mathring{\varepsilon}^{ijk}$
    \item 第二类克氏符号 $\mathring{\Gamma}^i_{jk}$
    \item 协变微分算子 $\mathring{\nabla}_i$ 和哈密顿算子 $\mathring{\nabla}$
\end{itemize}

对于瞬时构形,同样一些量或算子的记号分别为

\begin{itemize}
    \item 径矢 $\boldsymbol{r}$
    \item 协变基矢量 $\boldsymbol{e}_i = \frac{\partial \boldsymbol{r}}{\partial \xi^i}$ ,逆变基矢量 $\boldsymbol{e}_i$
    \item 度量张量的协变分量 $g_{ij}$ 和逆变分量 $g^{ij}$
    \item 置换张量的协变分量 $\varepsilon_{ijk}$ 和逆变分量 $\varepsilon^{ijk}$
    \item 第二类克氏符号 $\Gamma^i_{jk}$
    \item 协变微分算子 $\nabla_i$ 和哈密顿算子 $\nabla$
\end{itemize}

需要强调,对于不同的构形,逆变基矢量是由相应随体坐标系的度量张量逆变分量定义的,协变微分算子和哈密顿算子也有类似的区别。在计算时不能混淆。

\subsection{两个应变张量}

首先考虑物质线微元的变化,可以用一系列张量(统称为应变张量)来描述物质线微元长度的变化,也可以用一个张量(变形梯度张量)来直接描述物质线微元在变形前后的对应关系(相当于从初始时刻 $t_0$ 的物质线微元到时刻 $t$ 的物质线微元上的映射),这是描述变形的两种不同方法,前者只关注变形,后者还涉及关于转动的信息。我们分别讨论这两种方法,并建立两者之间的联系。

用径矢 $\mathring{\boldsymbol{r}}$ 的微分 $d \mathring{\boldsymbol{r}}$ 表示初始时刻的一段物质线微元,它在时刻 $t$ 移动到另一个位置,用径矢 $\boldsymbol{r}$ 的微分 $d \boldsymbol{r}$ 表示。我们认为物质线微元始终都是直线段,这在极限意义下是合理的。这两个径矢的终点(即这段物质线的起点)分别是拉格朗日坐标同为 $\xi^1, \xi^2, \xi^3$ 的同一个物质点在不同时刻所在的空间点,而该物质点在这段时间内的位移 $\boldsymbol{w}$ 正好是这两个径矢之差, $\boldsymbol{w} = \boldsymbol{r} - \mathring{\boldsymbol{r}}$ 。

由于变形,这段物质线微元会伸长或缩短,但其终点的拉格朗日坐标始终是 $(\xi^1 + d \xi^1, \xi^2 + d \xi^2, \xi^3 + d \xi^3)$ 。为了方便,我们来计算物质线微元 $d \mathring{\boldsymbol{r}}$ 长度平方的变化。根据径矢微分的分解式,可以写出:

\begin{gather*}
    d \mathring{\boldsymbol{r}}
    = \mathring{\boldsymbol{e}}_i d \xi^i \quad
    d \boldsymbol{r}
    = \boldsymbol{e}_i d \xi^i \\
    \Rightarrow
    |d \boldsymbol{r}|^2 - |d \mathring{\boldsymbol{r}}|^2
    = (g_{ij} - \mathring{g}_{ij}) d \xi^i d \xi^j
\end{gather*}

因为同一个物质点的拉格朗日坐标始终保持不变,所以逆变的量 $d \xi^i d \xi^j$ 是右边表达式中的公因子。又因为长度的平方是与坐标系无关的标量,所以这个等式表明 $g_{ij} - \mathring{g}_{ij}$ 必然是协变的(张量识别定理)。我们据此引入两个应变张量 $\mathring{\boldsymbol{\varepsilon}}$ 和 $\boldsymbol{\varepsilon}$ ,它们在随体坐标系下具有相同的协变分量 $\varepsilon_{ij}$ ,但分别是在初始构形和瞬时构形的随体坐标系中定义的:

\begin{gather*}
    \mathring{\boldsymbol{\varepsilon}}
    = \varepsilon_{ij} \mathring{\boldsymbol{e}}^i \mathring{\boldsymbol{e}}^j \\
    \boldsymbol{\varepsilon}
    = \varepsilon_{ij} \boldsymbol{e}^i \boldsymbol{e}^j \\
    \varepsilon_{ij}
    = \frac{1}{2} (g_{ij} - \mathring{g}_{ij})
\end{gather*}

它们显然都是对称的。

需要强调,这两个应变张量在随体坐标系下的协变分量相同,但是相应的逆变基矢量不同,并且各自角标的升降应当利用各自随体坐标系的度量张量分量来完成,不能混用。

\subsection{应变张量协变分量的物理意义}

应变张量在随体坐标系下的协变分量 $\varepsilon_{ij}$ 具有明显的物理意义,它们直接关系到变形的一些特征量。

首先考虑角标相同时的分量 $\varepsilon_{ii}$ ,可以写出:

$$
\varepsilon_{ii}
= \frac{1}{2} \mathring{g}_{ii} (\frac{|\boldsymbol{e}_i|^2}{|\mathring{\boldsymbol{e}}_i|^2} - 1)
$$

\begin{remark}
    这里显然不对 $i$ 求和,因为求和约定的条件不成立
\end{remark}

根据基矢量的定义和同一个物质点随体坐标的不变性,瞬时构形和初始构形的同一个物质点处的基矢量长度之比 $\frac{|\boldsymbol{e}_i|}{|\mathring{\boldsymbol{e}}_i|}$ 等于该物质点沿随体坐标线 $\xi^i$ 有同样的随体坐标增量 $\Delta \xi^i$ 时的相应径矢增量的长度之比 $\frac{|\Delta \boldsymbol{r}_i|}{|\Delta \mathring{\boldsymbol{r}}_i|}$ 在 $\Delta \xi^i$ 趋于零时的极限,而该长度之比正好等于 $l_i + 1$ ,其中 $l_i$ 是该物质点处沿随体坐标线 $\xi^i$ 的物质线微元发生变形后的伸长因数。于是,我们得到:

$$
\varepsilon_{ii}
= \frac{1}{2} \mathring{g}_{ii} ((l_i + 1)^2 - 1)
$$

这个关系式给出 $\varepsilon_{ii}$ 的物理意义,即 $\varepsilon_{ii}$ 与沿随体坐标线 $\xi^i$ 的物质线微元的伸长因数之间的定量关系。对于小变形,伸长因数是小量,这个关系式在精确到一阶小量时简化为:

$$
\varepsilon_{ii}
= \mathring{g}_{ii} l_i 
$$

即 $\varepsilon_{ii}$ 正比于伸长因数 $l_i$ 。

当初始构形的随体坐标系是笛卡儿直角坐标系时,进一步有:

$$
\varepsilon_{ii} = l_i 
$$

小变形的 $\varepsilon_{ii}$ 就是初始构形直角坐标轴上的物质线微元的伸长因数 $l_i$ 。

再考虑分量 $\varepsilon_{ij} \; (i \neq j)$ 的物理意义,设初始构形随体坐标系基矢量 $\mathring{\boldsymbol{e}}_i$ 与 $\mathring{\boldsymbol{e}}_j$ 之间的夹角为 $\mathring{\varphi}_{ij}$ ,瞬时构形随体坐标系基矢量 $\boldsymbol{e}_i$ 与 $\boldsymbol{e}_j$ 之间的夹角为 $\varphi_{ij}$ ,则:

$$
\varepsilon_{ij}
= \frac{1}{2} |\boldsymbol{e}_i| \cdot |\boldsymbol{e}_j| \cos \varphi_{ij} - \frac{1}{2} |\mathring{\boldsymbol{e}}_i| \cdot |\mathring{\boldsymbol{e}}_j| \cos \mathring{\varphi}_{ij}
$$

为简单起见,我们认为初始构形的随体坐标系是笛卡儿直角坐标系,则:

$$
\varepsilon_{ij}
= \frac{1}{2} |\boldsymbol{e}_i| \cdot |\boldsymbol{e}_j| \cos \varphi_{ij}
= \frac{1}{2} |\boldsymbol{e}_i| \cdot |\boldsymbol{e}_j| \sin \left( \frac{\pi}{2} - \varphi_{ij} \right)
$$

这个关系式给出 $\varepsilon_{ij}$ 的物理意义,即 $\varepsilon_{ij}$ 与瞬时构形随体坐标系基矢量 $\boldsymbol{e}_i, \boldsymbol{e}_j$ 的长度及夹角之间的定量关系。对于小变形,伸长因数和夹角的变化都是小量,这个关系式在精确到一阶小量时简化为:

$$
\varepsilon_{ij}
= \frac{1}{2} \left( \frac{\pi}{2} - \varphi_{ij} \right)
$$

对于一些简单的小变形,只要知道相应物质线的伸长因数和夹角变化,就能写出应变张量在随体坐标系下的协变分量 $\varepsilon_{ij}$ 。

\subsection{应变张量与位移的关系}

同一个物质点在不同构形下的径矢与该物质点的位移的关系为 $\boldsymbol{r} = \mathring{\boldsymbol{r}} + \boldsymbol{w}$ 。设 $\boldsymbol{w}$ 在初始构形随体坐标系中的分量为 $\mathring{w}^i, \mathring{w}_i$ ,则对上式求 $\xi^i$ 的偏导数,得:

$$
\boldsymbol{e}_i = \mathring{\boldsymbol{e}}_i + \mathring{\nabla}_i \mathring{w}^k \mathring{\boldsymbol{e}}_k
$$

进而可得瞬时构形随体坐标系中的度量张量分量:

\begin{align*}
    g_{ij}
    &= (\mathring{\boldsymbol{e}}_i + \mathring{\nabla}_i \mathring{w}^k \boldsymbol{e}_k) \cdot (\mathring{\boldsymbol{e}}_j + \mathring{\nabla}_j \mathring{w}^l \boldsymbol{e}_l) \\
    \Rightarrow
    g_{ij}
    &= \mathring{g}_{ij} + \mathring{\nabla}_i \mathring{w}_j + \mathring{\nabla}_j \mathring{w}_i + \mathring{\nabla}_i \mathring{w}^k \mathring{\nabla}_j \mathring{w}_k
\end{align*}

以及应变张量的协变分量

$$
\varepsilon_{ij}
= \frac{1}{2} (\mathring{\nabla}_i \mathring{w}_j + \mathring{\nabla}_j \mathring{w}_i + \mathring{\nabla}_i \mathring{w}^k \mathring{\nabla}_j \mathring{w}_k)
$$

所以

$$
\mathring{\boldsymbol{\varepsilon}}
= \frac{1}{2} (\mathring{\nabla} \boldsymbol{w} + ( \mathring{\nabla} \boldsymbol{w})^T + \mathring{\nabla} \boldsymbol{w} \cdot ( \mathring{\nabla} \boldsymbol{w})^T)
$$

类似地,设 $\boldsymbol{w}$ 在瞬时构形随体坐标系中的分量为 $w^i, w_i$ ,则:

$$
\varepsilon_{ij}
= \frac{1}{2} (\nabla_i w_j + \nabla_j w_i - \nabla_i w^k \nabla_j w_k)
$$

$$
\boldsymbol{\varepsilon}
= \frac{1}{2} (\nabla \boldsymbol{w} + (\nabla \boldsymbol{w})^T - \nabla \boldsymbol{w} \cdot ( \nabla \boldsymbol{w})^T)
$$

对于小变形,当位移梯度是小量(即当位移梯度在笛卡尔直角坐标系中的所有分量都是小量)时,忽略二阶小量后有:

$$
\varepsilon_{ij}
= \frac{1}{2} (\nabla_i w_j + \nabla_j w_i)
$$

$$
\boldsymbol{\varepsilon}
= \mathring{\boldsymbol{\varepsilon}}
= \frac{1}{2} (\nabla \boldsymbol{w} + (\nabla \boldsymbol{w})^T)
$$

从上述结果还可以看出应变张量的一个非同寻常的性质。一方面,应变张量是二阶对称张量,所以其独立分量的数目在一般情况下等于 $6$ 。另一方面,应变张量可以通过位移表示出来,所以应变张量只能有 $3$ 个独立分量。这意味着,应变张量的分量必然受到某种额外约束,而任意给出的二阶对称张量不一定对应着某种实际存在的变形。这样的约束是应变张量的分量所必须满足的一组微分方程,我们称之为应变的协调方程,其存在性是由初始构形与瞬时构形必须通过位移矢量相关联的条件决定的,我们将在\ref{sec:compatibilityEqu}节再讨论这个问题。

\subsection{变形的一些特征量}
\label{sec:featuresInTrans}

知道应变张量的协变分量 $\varepsilon_{ij}$ 后,就能计算变形的一些特征量。

考虑物质线微元的伸长因数,我们知道:

$$
l = \frac{|d \boldsymbol{r}|}{|d \mathring{\boldsymbol{r}}|} - 1
$$

又由 $\varepsilon_{ij}$ 的定义:

$$
|d \boldsymbol{r}|^2 - |d \mathring{\boldsymbol{r}}|^2 = 2 \varepsilon_{ij} d \xi^i d \xi^j
$$

$$
\Rightarrow
l
= \sqrt{1 + \frac{2 \varepsilon_{ij} d \xi^i d \xi^j}{|d \mathring{\boldsymbol{r}}|^2}} - 1
= \sqrt{1 + \frac{2 d \mathring{\boldsymbol{r}} \cdot \mathring{\boldsymbol{\varepsilon}} \cdot d \mathring{\boldsymbol{r}}}{|d \mathring{\boldsymbol{r}}|^2}} - 1
$$

取 $d \mathring{\boldsymbol{r}}$ 方向上的单位矢量 $\mathring{\boldsymbol{\alpha}} = \frac{d \mathring{\boldsymbol{r}}}{|d \mathring{\boldsymbol{r}}|}$ 及其逆变分量 $\mathring{\alpha}^i = \frac{d \xi^i}{|d \mathring{\boldsymbol{r}}|}$ ,则:

$$
l
= \sqrt{1 + 2 \varepsilon_{ij} \mathring{\alpha}^i \mathring{\alpha}^j} - 1
= \sqrt{1 + 2 \mathring{\boldsymbol{\alpha}} \cdot \mathring{\boldsymbol{\varepsilon}} \cdot \mathring{\boldsymbol{\alpha}}} - 1
$$

可以发现,对于给定的变形,一段物质线微元的伸长因数只与这段物质线在初始构形中的方向有关,而与其长度无关。

对于小变形:

$$
l
= \varepsilon_{ij} \mathring{\alpha}^i \mathring{\alpha}^j
= \mathring{\boldsymbol{\alpha}} \cdot \mathring{\boldsymbol{\varepsilon}} \cdot \mathring{\boldsymbol{\alpha}}
$$

接下来考虑物质线之间的夹角。设初始构形中的物质线微元 $d \mathring{\boldsymbol{r}}_{(1)}, \, d \mathring{\boldsymbol{r}}_{(2)}$ 具有共同的起点,相应的拉格朗日坐标为 $\xi^i$ ,它们在瞬时构形中分别对应 $d \boldsymbol{r}_{(1)} = \boldsymbol{e}_i d \xi^i_{(1)}, \, d \boldsymbol{r}_{(2)} = \boldsymbol{e}_i d \xi^i_{(2)}$ ,这两条物质线的夹角记为 $\psi$ ,则:

$$
d \boldsymbol{r}_{(1)} \cdot d \boldsymbol{r}_{(2)}
= |d \boldsymbol{r}_{(1)}| |d \boldsymbol{r}_{(2)}| \cos \psi
$$

$$
\Rightarrow
\cos \psi
= \frac{(\mathring{g}_{ij} + 2 \varepsilon_{ij}) d \xi^i_{(1)} d \xi^j_{(2)}}{\sqrt{|d \mathring{\boldsymbol{r}}_{(1)}|^2 + 2 \varepsilon_{ij} d \xi^i_{(1)} d \xi^j_{(1)}} \sqrt{|d \mathring{\boldsymbol{r}}_{(2)}|^2 + 2 \varepsilon_{ij} d \xi^i_{(2)} d \xi^j_{(2)}}}
$$

记 $\mathring{\alpha}^i_{(1)}, \, \mathring{\alpha}^i_{(2)}$ 为初始构形中相应物质线微元方向上的单位矢量的逆变分量,则:

$$
\cos \psi
= \frac{(\mathring{g}_{ij} + 2 \varepsilon_{ij}) \mathring{\alpha}^i_{(1)} \mathring{\alpha}^j_{(2)}}{\sqrt{1 + 2 \varepsilon_{ij} \mathring{\alpha}^i_{(1)} \mathring{\alpha}^j_{(1)}} \sqrt{1 + 2 \varepsilon_{ij} \mathring{\alpha}^i_{(2)} \mathring{\alpha}^j_{(2)}}}
$$

由此同样可见,对于给定的变形,物质线之间夹角的变化也只与初始构形中相应物质线微元的方向有关,而与其长度无关。
  
根据这个性质和物质线微元伸长因数的同样性质,连续介质微元的变形可以近似看做仿射变换,即让平直和平行性质保持不变的变换。
  
\begin{remark}
    具体而言,在运动保持连续的区域内,物质线微元在变形前后都近似于直线段微元,并且可以近似地认为相距足够近的彼此平行的两段物质线微元在变形后保持平行,其伸长因数是相等的,长度的比例也保持不变。因此,平行六面体微元在变形后仍然是平行六面体,平行四边形微元在变形后仍然是平行四边形。
\end{remark}
  
此外,利用微积分的思路容易证明,对于位于同一个位置的任意形状的物质线、面和体微元,相应的长度之比、面积之比和体积之比都是仿射变换下的不变量。我们可以利用物质微元变形的这种特性来进一步研究体积值和面积值的变化。

考虑任意形状的一个物质面微元。根据连续介质微元变形是仿射变换的结论,物质面微元面积积的相对变化与物质面微元的形状无关,所以我们可以在同样的位置另外考虑一个便于计算的平行四边形微元。其面积在初始构形和瞬时构形下分别可以用 $d \mathring{\boldsymbol{\Sigma}} = d \mathring{\boldsymbol{r}}_{(1)} \times d \mathring{ \boldsymbol{r}}_{(2)}$ 和 $d \boldsymbol{\Sigma} = d \boldsymbol{r}_{(1)} \times d \boldsymbol{r}_{(2)}$ 表示。

对同一个物质点处变形前后的物质线微元 $d \boldsymbol{r}, d\mathring{\boldsymbol{r}}$:

$$
d \boldsymbol{r}
= \boldsymbol{e}_j d \xi^j
= \frac{\partial \xi^i}{\partial x^j} \mathring{\boldsymbol{e}}_i d \xi^j
= \left( \frac{\partial \xi^i}{\partial x^j} \mathring{\boldsymbol{e}}_i \mathring{\boldsymbol{e}}^j \right) \cdot d \mathring{\boldsymbol{r}}
$$

记 $\boldsymbol{F} = \frac{\partial \xi^i}{\partial x^j} \mathring{\boldsymbol{e}}_i \mathring{\boldsymbol{e}}^j$ ,则:

$$
d \boldsymbol{r}
= \boldsymbol{F} \cdot d \mathring{\boldsymbol{r}}
$$

于是由 $Nanson$ 公式(详见作业三第4题):

\begin{gather*}
    d \boldsymbol{\Sigma}
    = (\boldsymbol{F} \cdot d \mathring{\boldsymbol{r}}_{(1)}) \times (\boldsymbol{F} \cdot d \mathring{\boldsymbol{r}}_{(2)})
    = (\det \boldsymbol{F}) \boldsymbol{F}^{-1T} \cdot (d \mathring{\boldsymbol{r}}_{(1)} \times d \mathring{ \boldsymbol{r}}_{(2)}) \\
    \Rightarrow
    d \boldsymbol{\Sigma}
    = (\det \boldsymbol{F}) \boldsymbol{F}^{-1T} \cdot d \mathring{\boldsymbol{\Sigma}} 
\end{gather*}

这里的 $\boldsymbol{F}$ 就是之后会出现的变形梯度张量。

再考虑任意形状的一个物质体微元,设其体积在初始构形和瞬时构形下分别是 $d \mathring{V}$ 和 $d V$ ,我们来研究体积的相对变化

$$
\theta = \frac{d V}{d \mathring{V}} - 1
$$
  
量 $\theta$ 称为物质体微元的体胀因数。

根据连续介质微元变形是仿射变换的结论,物质体微元体积的相对变化与物质体微元的形状无关,所以我们可以在同样的位置另外考虑一个便于计算的平行六面体微元。例如,可以在原物质体微元内某一点 $M$ 处作出以初始构形中随体坐标系三个基矢量 $\mathring{\boldsymbol{e}}_i$ 为棱的平行六面体 $\mathring{P}$ ,然后取它在点 $M$ 邻域内的一部分 $\mathring{P}'$ 作为上述平行六面体微元,使 $\mathring{P}'$ 同样以点 $M$ 为顶点并且与 $\mathring{P}$ 相似。
  
显然,平行六面体微元 $\mathring{P}'$ 在变形后是以瞬时构形中随体坐标系三个基矢量 $\boldsymbol{e}_i$ 为棱的平行六面体 $P$ 的一部分 $P'$ 。因此,平行六面体微元 $P', \mathring{P}'$ 的体积之比就是平行六面体 $P, \mathring{P}$ 的体积之比,而后两个平行六面体的体积恰好等于相应随体坐标系三个基矢量混合积的大小,其平方分别等于由相应随体坐标系度量张量协变分量所组成的矩阵的行列式 $\mathring{g} = \det (\mathring{g}_{ij})$  和 $g = \det (g_{ij})$ 。于是:
  
$$
\theta = \sqrt{\frac{g}{\mathring{g}}} - 1
$$

$$
\Rightarrow
\theta
= \sqrt{\det (\mathring{g}_{ij} + 2 \varepsilon_{ij}) \det (\mathring{g}^{ij})}
= \sqrt{\det (\delta^i_j + 2 \mathring{\varepsilon}^{i \cdot}_{\cdot j})} - 1
$$
  
容易验证,这个公式可以改写为

$$
\theta = \sqrt{1 + 2 \mathring{I}_1 + 4 \mathring{I}_2 + 8 \mathring{I}_3} - 1
$$
  
其中 $\mathring{I}_1, \mathring{I}_2, \mathring{I}_3$ 是应变张量 $\mathring{\boldsymbol{\varepsilon}}$ 的第一、第二、第三数值不变量。

\subsection{应变张量的主轴和主分量}

我们知道 $\mathring{\boldsymbol{\varepsilon}}, \boldsymbol{\varepsilon}$ 是对称二阶张量。由高等线性代数知识,对任意对称二阶张量 $\boldsymbol{T}$ ,存在笛卡尔直角坐标系 $y^i$ ,使得:

$$
\boldsymbol{T}
=  T_i \boldsymbol{e}^i
$$

其中 $\boldsymbol{e}^i$ 分别是坐标系 $y^i$ 的基矢量。

坐标系 $y^i$ 称为主坐标系, $T_i$ 称为主分量(或主值)。

要求主坐标系,只需考虑以下方程(其中 $\boldsymbol{a}$ 为一阶张量, $\lambda$ 为实数):

\begin{align*}
    &\boldsymbol{T} \cdot \boldsymbol{a}
    = \lambda \boldsymbol{a} \\
    \Rightarrow
    &(T^{i \cdot}_{\cdot j} - \lambda \delta^i_j) a^j \boldsymbol{e}_i = 0 \\
    \Rightarrow
    &\det (T^{i \cdot}_{\cdot j} - \lambda \delta^i_j) = 0
\end{align*}

上方程也可以写为:

$$
\lambda^3 - I_1 \lambda^2 + I_2 \lambda - I_3 = 0
$$

由上便可解出 $\lambda_i = T_i$ ,并求出主坐标系的基矢量 $\boldsymbol{a}^i$

显然:

$$
\begin{cases}
    I_1 = T_1 + T_2 + T_3 \\
    I_2 = T_1 T_2 + T_2 T_3 + T_3 T_1 \\
    I_3 = T_1 T_2 T_3
\end{cases}
$$

取 $\mathring{\boldsymbol{\varepsilon}}$ 的主坐标系(一般是局部的)为初始构形的拉格朗日坐标 $\xi^i$ ,则:

$$
\mathring{\boldsymbol{\varepsilon}}
= \sum_i \mathring{\varepsilon}_i \mathring{\boldsymbol{e}}^i \mathring{\boldsymbol{e}}^i
$$

这意味着

$$
\varepsilon_{ii} = \mathring{\varepsilon}_i = \frac{1}{2} ((1 + l_i)^2 - 1) \quad
\varepsilon_{ij} = 0 \; (i \neq j)
$$

所以局部变形可以理解为沿着 $\mathring{\varepsilon}$ 主轴方向的伸缩与 $\mathring{\varepsilon}$ 主坐标系整体的转动的叠加。

\begin{remark}
    局部运动=平移+转动+纯变形
\end{remark}

同时还可以得到: $g_{ij} = 0 \; (i \neq j)$ ,所以瞬时构形的随体坐标系是正交的。

接下来考虑 $\boldsymbol{\varepsilon}$ 的主值,我们有:

$$
\boldsymbol{\varepsilon}
= \sum_i \varepsilon_{ii} \boldsymbol{e}^i \boldsymbol{e}^i
$$

$g_{ij} = 0 \; (i \neq j)$ ,所以瞬时构形的随体坐标系是正交的。要得到主坐标系,只需对基矢量作归一化,得到:

$$
\boldsymbol{\varepsilon}
= \sum_i \varepsilon_{ii} g^{ii} \frac{\boldsymbol{e}^i}{|\boldsymbol{e}^i|} \frac{\boldsymbol{e}^i}{|\boldsymbol{e}^i|}
$$

上式中 $\frac{\boldsymbol{e}^i}{|\boldsymbol{e}^i|}$ 即为主坐标系的基矢量, $\varepsilon_i = \varepsilon_{ii} g^{ii}$ 为 $\boldsymbol{\varepsilon}$ 的主值。

由于 $g_{ij} = 0 \; (i \neq j)$ ,故由对角矩阵的逆的性质:

$$
g^{ii}
= (g_{ii})^{-1}
= \frac{1}{(1 + l_i)^2}
$$

$$
\Rightarrow
\boldsymbol{\varepsilon}
= \frac{1}{2} \left( 1 - \frac{1}{(1 + l_i)^2} \right) \frac{\boldsymbol{e}^i}{|\boldsymbol{e}^i|} \frac{\boldsymbol{e}^i}{|\boldsymbol{e}^i|}
$$

这样我们就得到了 $\boldsymbol{\varepsilon}$ 的主值 $\varepsilon_i$ :

$$
\varepsilon_i = \frac{1}{2} \left( 1 - \frac{1}{(1 + l_i)^2} \right)
$$

显然 $\mathring{\varepsilon}_i$ 与 $\varepsilon_i$ 有如下关系:

$$
\varepsilon_i
= \frac{\mathring{\varepsilon}_i}{1 + 2\mathring{\varepsilon}_i} \quad
\mathring{\varepsilon}_i
= \frac{\varepsilon_i}{1 - 2\varepsilon_i}
$$

变形为小变形时,保留一阶小量则有:

$$
\varepsilon_i = \mathring{\varepsilon}_i
$$

\subsection{变形梯度张量}

下面介绍另一个描述变形的张量。

对同一个物质点处变形前后的物质线微元 $d \boldsymbol{r}, d\mathring{\boldsymbol{r}}$ ,考虑这两个微元之间的关系:

$$
d \boldsymbol{r}
= \boldsymbol{e}_j d \xi^j
= (\boldsymbol{e}_i \mathring{\boldsymbol{e}}^i) \cdot (\mathring{\boldsymbol{e}}_j d \xi^j)
= (\boldsymbol{e}_i \mathring{\boldsymbol{e}}^i) \cdot d \mathring{\boldsymbol{r}}
$$

记 $\boldsymbol{F} = \boldsymbol{e}_i \mathring{\boldsymbol{e}}^i$ ,称之为变形梯度张量,则:

$$
d \boldsymbol{r}
= \boldsymbol{F} \cdot d \mathring{\boldsymbol{r}}
$$

变形梯度张量的“梯度”由下式体现:

$$
\boldsymbol{F}
= \nabla_i \boldsymbol{r} \mathring{\boldsymbol{e}}^i
= (\mathring{\nabla} \boldsymbol{r})^T
$$

由其定义,可以得到:

\begin{gather*}
    d \boldsymbol{r}
    = \boldsymbol{F} \cdot d \mathring{\boldsymbol{r}} \quad
    d \mathring{\boldsymbol{r}}
    = \boldsymbol{F}^{-1} \cdot d \boldsymbol{r} \\
    \boldsymbol{e}_i
    = \boldsymbol{F} \cdot \mathring{\boldsymbol{e}}_i \quad
    \mathring{\boldsymbol{e}}_i
    = \boldsymbol{F}^{-1} \cdot \boldsymbol{e}_i \\
    \boldsymbol{e}^i
    = \boldsymbol{F}^{-1T} \cdot \mathring{\boldsymbol{e}}^i \quad
    \mathring{\boldsymbol{e}}^i
    = \boldsymbol{F}^T \cdot \boldsymbol{e}^i
\end{gather*}

下面来看变形前后物质面微元、物质体微元之间的关系,

首先考察物质体微元:

$$
d V
= d \boldsymbol{r}_{(1)} \cdot (d \boldsymbol{r}_{(2)} \times d \boldsymbol{r}_{(3)})
= (\boldsymbol{F} \cdot d \mathring{\boldsymbol{r}}_{(1)}) \cdot [(\boldsymbol{F} \cdot d \mathring{\boldsymbol{r}}_{(2)}) \times (\boldsymbol{F} \cdot d \mathring{\boldsymbol{r}}_{(3)})]
$$

借助下面的公式:

\begin{remark}
    对任意非退化二阶张量 $\boldsymbol{F}$ ,一阶张量 $\boldsymbol{a}, \boldsymbol{b}, \boldsymbol{c}$ :

    $$
    (\boldsymbol{F} \cdot \boldsymbol{a}) \cdot [(\boldsymbol{F} \cdot \boldsymbol{b}) \times (\boldsymbol{F} \cdot \boldsymbol{c})]
    = \det \boldsymbol{F} [\boldsymbol{a} \cdot (\boldsymbol{b} \times \boldsymbol{c})]
    $$
\end{remark}

\begin{proof}
    \begin{align*}
        & (\boldsymbol{F} \cdot \boldsymbol{a}) \cdot [(\boldsymbol{F} \cdot \boldsymbol{b}) \times (\boldsymbol{F} \cdot \boldsymbol{c})] \\
        &= \varepsilon_{ijk} F^{i \cdot}_{\cdot l} F^{j \cdot}_{\cdot m} F^{k \cdot}_{\cdot n} a^l b^m c^n \\
        &= \frac{1}{6} \varepsilon_{ijk} F^{i \cdot}_{\cdot l} F^{j \cdot}_{\cdot m} F^{k \cdot}_{\cdot n} (a^l b^m c^n + a^m b^n c^l + a^n b^l c^m - a^l b^n c^m + a^m b^l c^n + a^n b^m c^l) \\
        &= (\frac{1}{6} \varepsilon_{ijk} \varepsilon^{lmn} F^{i \cdot}_{\cdot l} F^{j \cdot}_{\cdot m} F^{k \cdot}_{\cdot n}) (\varepsilon_{pqr} a^p b^q c^r) \\
        &= \det \boldsymbol{F} [\boldsymbol{a} \cdot (\boldsymbol{b} \times \boldsymbol{c})]
    \end{align*}

    上式也可以使用 $Nanson$ 公式证明:

    \begin{align*}
        & (\boldsymbol{F} \cdot \boldsymbol{a}) \cdot [(\boldsymbol{F} \cdot \boldsymbol{b}) \times (\boldsymbol{F} \cdot \boldsymbol{c})] \\
        &= (\boldsymbol{F} \cdot \boldsymbol{a}) \cdot [(\det \boldsymbol{F}) \boldsymbol{F}^{-1T} \cdot (\boldsymbol{b} \times \boldsymbol{c})] \\
        &= (\det \boldsymbol{F}) \boldsymbol{F} \cdot \boldsymbol{a} \cdot \boldsymbol{F}^{-1T} \cdot (\boldsymbol{b} \times \boldsymbol{c}) \\
        &= (\det \boldsymbol{F}) \boldsymbol{F} \cdot \boldsymbol{F}^{-1} \cdot \boldsymbol{a} \cdot (\boldsymbol{b} \times \boldsymbol{c}) \\
        &= \det \boldsymbol{F} [\boldsymbol{a} \cdot (\boldsymbol{b} \times \boldsymbol{c})]
    \end{align*}
\end{proof}

可得:

$$
d V = \det \boldsymbol{F} \; d \mathring{V}
$$

结合\ref{sec:featuresInTrans}节中得到的式子 $d V = \sqrt{\frac{g}{\mathring{g}}} d \mathring{V}$ ,还能得到下式:

$$
\det \boldsymbol{F} = \sqrt{\frac{g}{\mathring{g}}}
$$

\begin{remark}
    也可以用数学方法证明上式。设 $\boldsymbol{e}_i$ 在笛卡尔坐标系下的分量为 $e_{ix}, e_{iy}, e_{iz}$ , $\boldsymbol{F}$ 在初始构形拉格朗日坐标系下的混合分量为 $\mathring{F}^{i \cdot}_{\cdot j}$ ,则:

    $$
    \boldsymbol{e}_j
    = \boldsymbol{F} \cdot \mathring{\boldsymbol{e}}_j
    = \mathring{F}^{i \cdot}_{\cdot j} \mathring{\boldsymbol{e}}_i
    $$

    $$
    \Rightarrow
    \begin{vmatrix}
        e_{1x} & e_{1y} & e_{1z} \\
        e_{2x} & e_{2y} & e_{2z} \\
        e_{3x} & e_{3y} & e_{3z}
    \end{vmatrix}
    = \det \boldsymbol{F^{T}}
    \begin{vmatrix}
        \mathring{e}_{1x} & \mathring{e}_{1y} & \mathring{e}_{1z} \\
        \mathring{e}_{2x} & \mathring{e}_{2y} & \mathring{e}_{2z} \\
        \mathring{e}_{3x} & \mathring{e}_{3y} & \mathring{e}_{3z} \\
    \end{vmatrix}
    = \det \boldsymbol{F}
    \begin{vmatrix}
        \mathring{e}_{1x} & \mathring{e}_{1y} & \mathring{e}_{1z} \\
        \mathring{e}_{2x} & \mathring{e}_{2y} & \mathring{e}_{2z} \\
        \mathring{e}_{3x} & \mathring{e}_{3y} & \mathring{e}_{3z} \\
    \end{vmatrix}
    $$

    令上式两边对应的矩阵分别为 $E, \mathring{E}$ 。则可以将上式写作:

    $$
    \det E = \det \boldsymbol{F} \det \mathring{E}
    $$

    于是:

    $$
    g
    = \det (E E^T)
    = (\det E)^2 \\
    = (\det \boldsymbol{F})^2 (\det \mathring{E})^2
    = (\det \boldsymbol{F})^2 \mathring{g}
    $$

    $$
    \Rightarrow
    \det \boldsymbol{F} = \sqrt{\frac{g}{\mathring{g}}}
    $$
\end{remark}

容易发现 $\det \boldsymbol{F} > 0$ ,在后面的讨论中将得到此式的物理意义。

然后考察物质面微元:

$$
d \boldsymbol{\Sigma}
= d \boldsymbol{r}_{(1)} \times d \boldsymbol{r}_{(2)}
= \varepsilon_{ijk} d \xi^i_{(1)} d \xi^j_{(2)} \boldsymbol{e}^k
$$

结合 $\boldsymbol{e}^k = \boldsymbol{F}^{-1T} \mathring{\boldsymbol{e}}^k$ 得到:

$$
d \boldsymbol{\Sigma}
= \varepsilon_{ijk} d \xi^i_{(1)} d \xi^j_{(2)} \boldsymbol{F}^{-1T} \mathring{\boldsymbol{e}}^k
$$

$i, j, k$ 中至少有两个相同时, $\varepsilon_{ijk} = \mathring{\varepsilon}_{ijk} = 0$ ; $i, j, k$ 互不相同时:

$$
\frac{\varepsilon_{ijk}}{\mathring{\varepsilon}_{ijk}}
= \pm \sqrt{\frac{g}{\mathring{g}}}
= \pm \det \boldsymbol{F}
$$

我们已经得到过 $d \boldsymbol{\Sigma} = (\det \boldsymbol{F}) \boldsymbol{F}^{-1T} \cdot d \mathring{\boldsymbol{\Sigma}}$ ,所以上式只能取正号。

综上所述,对任意的 $i, j, k$ ,有:

$$
\frac{\varepsilon_{ijk}}{\sqrt{g}}
= \frac{\mathring{\varepsilon}_{ijk}}{\sqrt{\mathring{g}}} \quad
或 \quad
\varepsilon_{ijk}
= (\det \boldsymbol{F}) \; \mathring{\varepsilon}_{ijk}
$$

这个式子说明变形前后随体坐标系的手性不变。同时还可以知道,变形前后物质面微元的取向也是不变的。镜像变换式的变形是不可能发生的。

如果将 $\boldsymbol{F}$ 视为一种局部线性变换,则:

$$
d \boldsymbol{r}
= \boldsymbol{F} \cdot d \mathring{\boldsymbol{r}}
= \mathring{F}^{i \cdot}_{\cdot j} d \xi^j \mathring{\boldsymbol{e}}_i
$$

$$
\Rightarrow
d x^i = \mathring{F}^{i \cdot}_{\cdot j} d \xi^j
$$

这个变换是局部仿射变换。

\begin{remark}
    保持平行 保持比例 顺序重要(小变形时可交换)
\end{remark}

下面来看变形梯度张量的极分解。由高等线性代数知识, $\boldsymbol{F}$ 可以分解如下:

$$
\boldsymbol{F}
= \boldsymbol{R} \cdot \boldsymbol{U}
= \boldsymbol{V} \cdot \boldsymbol{R}
$$

其中 $\boldsymbol{R}$ 是正交张量(即满足 $\boldsymbol{R}^{-1} = \boldsymbol{R}^T$ 的张量),描述物质的转动; $\boldsymbol{U}, \boldsymbol{V}$ 是正定的二阶对称张量(即满足: $\forall \boldsymbol{a}, \; \boldsymbol{a} \cdot \boldsymbol{U} \cdot \boldsymbol{a} > 0$ 的张量),描述物质的纯变形。

这样分解表明物质的变形可以分为两步,这两步可以是先纯变形后旋转,也可以是先旋转后纯变形。

\begin{remark}
    由 $\det \boldsymbol{F} > 0$ 可以得到:

    $$
    \det \boldsymbol{R} = 1
    $$
\end{remark}

显然对 $\boldsymbol{U}, \boldsymbol{V}$ 均可求出主坐标系。记 $\boldsymbol{U} = \sum_i \lambda_i \boldsymbol{n}_i \boldsymbol{n}_i$ ,则:

$$
\boldsymbol{F}
= \sum_i \lambda_i \boldsymbol{N}_i \boldsymbol{n}_i
$$

$$
\boldsymbol{V} = \sum_i \lambda_i \boldsymbol{N}_i \boldsymbol{N}_i
$$

其中 $\boldsymbol{N}_i = \boldsymbol{R} \cdot \boldsymbol{n}_i$ ,显然是单位矢量。

最后,考虑变形梯度张量与位移、应变张量的关系。

对于位移:

$$
\boldsymbol{F}
= \nabla_i \boldsymbol{r} \mathring{\boldsymbol{e}}^i
= \nabla_i (\mathring{\boldsymbol{r}} + \boldsymbol{w}) \mathring{\boldsymbol{e}}^i
= \boldsymbol{g} + (\mathring{\nabla} \boldsymbol{w})^T
$$

$$
\boldsymbol{F}^{-1}
= \nabla_i \boldsymbol{\mathring{\boldsymbol{r}}} \boldsymbol{e}^i
= \nabla_i (\boldsymbol{r} - \boldsymbol{w}) \boldsymbol{e}^i
= \boldsymbol{g} - (\nabla \boldsymbol{w})^T
$$

对于应变张量:

$$
\varepsilon_{ij}
= \frac{1}{2} (\boldsymbol{e}_i \cdot \boldsymbol{e}_j - \mathring{\boldsymbol{e}}_i \cdot \mathring{\boldsymbol{e}}_j)
$$

而

\begin{gather*}
    \boldsymbol{e}_i \cdot \boldsymbol{e}_j
    = \boldsymbol{F} \cdot \mathring{\boldsymbol{e}}_i \cdot \boldsymbol{F} \cdot \mathring{\boldsymbol{e}}_j
    = \mathring{\boldsymbol{e}}_i \cdot (\boldsymbol{F}^T \cdot \boldsymbol{F}) \cdot \mathring{\boldsymbol{e}}_j \\
    \mathring{\boldsymbol{e}}_i \cdot \mathring{\boldsymbol{e}}_j
    = \boldsymbol{F}^{-1} \cdot \boldsymbol{e}_i \cdot \boldsymbol{F}^{-1} \cdot \boldsymbol{e}_j
    = \boldsymbol{e}_i \cdot (\boldsymbol{F}^{-1T}\cdot \boldsymbol{F}^{-1}) \cdot \boldsymbol{e}_j
\end{gather*}

\begin{gather*}
    \Rightarrow
    \varepsilon_{ij}
    = \frac{1}{2} \mathring{\boldsymbol{e}}_i \cdot (\boldsymbol{F}^T \cdot \boldsymbol{F} - \boldsymbol{g}) \cdot \mathring{\boldsymbol{e}}_j \\
    \varepsilon_{ij}
    = \frac{1}{2} \boldsymbol{e}_i \cdot (\boldsymbol{g} - \boldsymbol{F}^{-1T}\cdot \boldsymbol{F}^{-1}) \cdot \boldsymbol{e}_j
\end{gather*}

\begin{gather*}
    \Rightarrow
    \boldsymbol{F}^T \cdot \boldsymbol{F}
    = \boldsymbol{g} + 2 \mathring{\boldsymbol{\varepsilon}} \\
    \boldsymbol{F}^{-1T}\cdot \boldsymbol{F}^{-1}
    = \boldsymbol{g} - 2 \boldsymbol{\varepsilon}
\end{gather*}

\subsection{变形随时间的变化率}

接下来我们开始分析变形与时间的关系,首先我们考虑:

$$
e_{ij}
= \frac{\partial}{\partial t} \varepsilon_{ij} (\xi^\alpha, t)
$$

初始构形选定,不随时间变化,所以:

$$
2 e_{ij}
= \frac{\partial}{\partial t} g_{ij} (\xi^\alpha, t)
= \frac{\partial \boldsymbol{e}_i}{\partial t} 
\cdot \boldsymbol{e}_j + \boldsymbol{e}_i \cdot \frac{\partial \boldsymbol{e}_j}{\partial t}
$$

而

$$
\frac{\partial \boldsymbol{e}_i}{\partial t}
= \frac{\partial^2 \boldsymbol{r}}{\partial \xi^i \partial t}
= \frac{\partial \boldsymbol{v}}{\partial \xi^i}
= \nabla_i \boldsymbol{v}
$$

$$
\Rightarrow
e_{ij}
=\frac{1}{2} (\nabla_i v_j + \nabla_j v_i)
$$

于是可以定义应变率张量:

$$
\boldsymbol{e}
= \frac{1}{2} (\nabla \boldsymbol{v} + (\nabla \boldsymbol{v})^T)
$$

\begin{remark}
    可以证明:

    $$
    e^{i \cdot}_{\cdot i} = \nabla \cdot \boldsymbol{v}
    $$
\end{remark}

然后,考虑基矢量随时间的变化率。协变基矢量的变化率如上,故只需考虑逆变基矢量:

\begin{gather*}
    \frac{\partial \boldsymbol{e}_i}{\partial t}
    = \nabla_i v_j \boldsymbol{e}^j
    = g_{ij} \frac{\partial \boldsymbol{e}^j}{\partial t} + (\nabla_i v_j + \nabla_j v_i) \boldsymbol{e}^j \\
    \Rightarrow
    g_{ij} \frac{\partial \boldsymbol{e}^j}{\partial t}
    = - \nabla_j v_i \boldsymbol{e}^j \\
    \Rightarrow
    g^{ik} g_{ij} \frac{\partial \boldsymbol{e}^j}{\partial t}
    = \delta^k_j \frac{\partial \boldsymbol{e}^j}{\partial t}
    = \frac{\partial \boldsymbol{e}^k}{\partial t}
    = - g^{ik} \nabla_j v_i \boldsymbol{e}^j
    = - \nabla_j v^k \boldsymbol{e}^j
\end{gather*}

即

$$
\frac{\partial \boldsymbol{e}^k}{\partial t}
= - \nabla v^k
$$

因此我们得到以下关系式:

$$
\frac{\partial \boldsymbol{e}_i}{\partial t}
= \nabla_i \boldsymbol{v}
\qquad
\frac{\partial \boldsymbol{e}^i}{\partial t}
= - \nabla v^i
$$

由此可以得到变形梯度张量随时间的变化率:

$$
\frac{\partial \boldsymbol{F}}{\partial t}
= (\mathring{\nabla} \boldsymbol{v})^T
= (\nabla \boldsymbol{v})^T \cdot \boldsymbol{F}
$$

$$
\frac{\partial \boldsymbol{F}^{-1}}{\partial t}
= - \boldsymbol{F}^{-1} \cdot (\nabla \boldsymbol{v})^T
$$

接下来看物质微元对时间的变化率:

$$
\frac{\partial}{\partial t} d \boldsymbol{r}
= \frac{\partial \boldsymbol{e}_i}{\partial t} d \xi^i
=  \nabla_i \boldsymbol{v} d \xi^i
= d \boldsymbol{r} \cdot \nabla \boldsymbol{v}
= (\nabla \boldsymbol{v})^T \cdot d \boldsymbol{r}
$$

$$
\frac{\partial}{\partial t} |d \boldsymbol{r}|^2
= \frac{\partial}{\partial t} d \boldsymbol{r} \cdot d \boldsymbol{r} + d \boldsymbol{r} \cdot \frac{\partial}{\partial t} d \boldsymbol{r}
= 2 d \boldsymbol{r} \cdot \boldsymbol{e} \cdot d \boldsymbol{r}
$$

借助公式

$$
\frac{\partial}{\partial t} \det \boldsymbol{F}
= \det \boldsymbol{F} (\nabla \cdot \boldsymbol{v})
$$

\begin{proof}
    由 $\det \boldsymbol{F} \det \mathring{E} = \det E$ ,注意到 $E$ 是 $\boldsymbol{e}_1, \boldsymbol{e}_2, \boldsymbol{e}_3$ 的混合积:

    $$
    (\det \mathring{E}) \frac{\partial}{\partial t} \det \boldsymbol{F}
    = \frac{\partial}{\partial t} [\boldsymbol{e}_1 \cdot (\boldsymbol{e}_2 \times \boldsymbol{e}_3)]
    $$

    而

    $$
    \frac{\partial}{\partial t} [\boldsymbol{e}_1 \cdot (\boldsymbol{e}_2 \times \boldsymbol{e}_3)]
    = \nabla_1 \boldsymbol{v} \cdot (\boldsymbol{e}_2 \times \boldsymbol{e}_3) + \nabla_2 \boldsymbol{v} \cdot (\boldsymbol{e}_3 \times \boldsymbol{e}_1) + \nabla_3 \boldsymbol{v} \cdot (\boldsymbol{e}_1 \times \boldsymbol{e}_2) \\
    $$

    对第一项:

    $$
    \nabla_1 \boldsymbol{v} \cdot (\boldsymbol{e}_2 \times \boldsymbol{e}_3)
    = \nabla_1 v^j \boldsymbol{e}_j \cdot (\boldsymbol{e}_2 \times \boldsymbol{e}_3)
    = (\nabla_1 v^1) [\boldsymbol{e}_1 \cdot (\boldsymbol{e}_2 \times \boldsymbol{e}_3)]
    $$

    同理:

    \begin{align*}
        \nabla_2 \boldsymbol{v} \cdot (\boldsymbol{e}_3 \times \boldsymbol{e}_1)
        &= (\nabla_2 v^2) [\boldsymbol{e}_1 \cdot (\boldsymbol{e}_2 \times \boldsymbol{e}_3)] \\
        \nabla_3 \boldsymbol{v} \cdot (\boldsymbol{e}_1 \times \boldsymbol{e}_2)
        &= (\nabla_3 v^3) [\boldsymbol{e}_1 \cdot (\boldsymbol{e}_2 \times \boldsymbol{e}_3)]
    \end{align*}

    故:

    $$
    \frac{\partial}{\partial t} [\boldsymbol{e}_1 \cdot (\boldsymbol{e}_2 \times \boldsymbol{e}_3)]
    = (\nabla \cdot \boldsymbol{v}) [\boldsymbol{e}_1 \cdot (\boldsymbol{e}_2 \times \boldsymbol{e}_3)]
    $$

    $$
    \Rightarrow
    \frac{\partial}{\partial t} \det \boldsymbol{F}
    = \det \boldsymbol{F} (\nabla \cdot \boldsymbol{v})
    $$
\end{proof}

\begin{remark}
    从上面的证明过程能够得到:

    $$
    \frac{\partial \varepsilon_{ijk}}{\partial t}
    = (\nabla \cdot \boldsymbol{v}) \varepsilon_{ijk}
    $$
\end{remark}

还能得到:

$$
\frac{\partial}{\partial t} d \boldsymbol{\Sigma}
= (\nabla \cdot \boldsymbol{v}) d \boldsymbol{\Sigma} - \nabla \boldsymbol{v} \cdot d \boldsymbol{\Sigma}
$$

$$
\frac{\partial}{\partial t} d V
= \nabla \cdot \boldsymbol{v} d V
$$

上面的两个式子也可以用局部变形的仿射性推出。将 $d \boldsymbol{\Sigma}, d V$ 分别写成:

\begin{align*}
    d \boldsymbol{\Sigma}
    &= d \boldsymbol{r}_{(1)} \times d \boldsymbol{r}_{(2)}
    = \varepsilon_{ijk} d \xi^i_{(1)} d \xi^j_{(2)} \boldsymbol{e}^k \\
    d V
    &= d \boldsymbol{r}_{(1)} \cdot (d \boldsymbol{r}_{(2)} \times d \boldsymbol{r}_{(3)})
    = \varepsilon_{ijk} d \xi^i_{(1)} d \xi^j_{(2)} d \xi^k_{(3)}   
\end{align*}

再作相关运算即可。

\begin{remark}
    此外,还可以得到物质面微元单位矢量随时间的变化率。由 $d \boldsymbol{\Sigma} = |d \boldsymbol{\Sigma}| \boldsymbol{n} = d \Sigma \; \boldsymbol{n}$ 可得:

    $$
    d \Sigma \frac{\partial \boldsymbol{n}}{\partial t} + \frac{\partial d \Sigma}{\partial t} \boldsymbol{n}
    = (\nabla \cdot \boldsymbol{v}) d \boldsymbol{\Sigma} - \nabla \boldsymbol{v} \cdot d \boldsymbol{\Sigma}
    $$

    而

    $$
    d \Sigma^2 = d \boldsymbol{\Sigma} \cdot d \boldsymbol{\Sigma}
    $$

    $$
    \Rightarrow
    2 \; d \Sigma \; \frac{\partial d \Sigma}{\partial t}
    = 2 d \boldsymbol{\Sigma} \cdot [(\nabla \cdot \boldsymbol{v}) d \boldsymbol{\Sigma} - \nabla \boldsymbol{v} \cdot d \boldsymbol{\Sigma}]
    $$

    $$
    \Rightarrow
    \frac{\partial d \Sigma}{\partial t}
    = (\nabla \cdot \boldsymbol{v}) d \Sigma - \frac{d \boldsymbol{\Sigma} \cdot \nabla \boldsymbol{v} \cdot d \boldsymbol{\Sigma}}{d \Sigma}
    $$

    于是可以得到:

    $$
    \frac{\partial \boldsymbol{n}}{\partial t}
    = (\boldsymbol{n} \cdot \nabla \boldsymbol{v} \cdot \boldsymbol{n}) \boldsymbol{n} - \nabla \boldsymbol{v} \cdot \boldsymbol{n}
    $$
\end{remark}

\subsection{速度分布}

研究同一时刻速度的分布:

$$
\boldsymbol{v}(\boldsymbol{r} + d \boldsymbol{r}) - \boldsymbol{v}(\boldsymbol{r})
= d \boldsymbol{v}
= d \boldsymbol{r} \cdot \nabla \boldsymbol{v}
$$

将 $\nabla \boldsymbol{v}$ 拆成对称与反对称的两部分:

$$
\nabla \boldsymbol{v}
= \frac{1}{2} [\nabla \boldsymbol{v} + (\nabla \boldsymbol{v})^T] + \frac{1}{2} [\nabla \boldsymbol{v} - (\nabla \boldsymbol{v})^T]
$$

结合公式 $(\nabla \times \boldsymbol{a}) \times \boldsymbol{c} = \boldsymbol{c} \cdot \nabla \boldsymbol{a} - \boldsymbol{c} \cdot (\nabla \boldsymbol{a})^T$ ,便可得到:

$$
\boldsymbol{v}(\boldsymbol{r} + d \boldsymbol{r}) - \boldsymbol{v}(\boldsymbol{r})
= d \boldsymbol{r} \cdot \boldsymbol{e} + \frac{1}{2} (\nabla \times \boldsymbol{v}) \times d \boldsymbol{r}
$$

上式中 $\nabla \times \boldsymbol{v}$ 即为涡量 $\boldsymbol{\omega}$ (部分文献也将涡量定义为 $\frac{1}{2} \nabla \times \boldsymbol{v}$ ) ,它表示运动的局部角速度。 $d \boldsymbol{r} \cdot \boldsymbol{e}$ 为变形速度,是由变形引起的。

\subsection{协调方程}
\label{sec:compatibilityEqu}

之前提到过, $\mathring{\boldsymbol{\varepsilon}}, \boldsymbol{\varepsilon}, \boldsymbol{e}$ 等张量受到协调方程约束,下面试着求出这样的方程。

以有势速度为例,假想速度 $\boldsymbol{v}$ 满足:

$$
\exists \varphi, \; s.t. \; \;
\boldsymbol{v} = \nabla \varphi
$$

那么由 $\nabla \times \nabla \varphi = 0$ :

$$
\nabla \times \boldsymbol{v} = 0
$$

受此启发,我们来求 $\boldsymbol{e}$ 的协调方程,由于:

$$
\boldsymbol{e} = \frac{1}{2} (\nabla \boldsymbol{v} + (\nabla \boldsymbol{v})^T)
$$

参照公式 $(\nabla \times \boldsymbol{a}) \times \boldsymbol{c} = \boldsymbol{c} \cdot \nabla \boldsymbol{a} - \boldsymbol{c} \cdot (\nabla \boldsymbol{a})^T$ ,由于任意张量与度量张量的点积等于其自身,故计算 $(\nabla \times \boldsymbol{v}) \times \boldsymbol{g}$ ,得到:

\begin{gather*}
    (\nabla \times \boldsymbol{v}) \times \boldsymbol{g}
    = (\nabla \boldsymbol{v})^T - \nabla \boldsymbol{v} \\
    \Rightarrow
    \boldsymbol{e}
    = \frac{1}{2} (\nabla \times \boldsymbol{v}) \times \boldsymbol{g} + \nabla \boldsymbol{v}
\end{gather*}

记 $\boldsymbol{a} = \nabla \times \boldsymbol{v}$ ,利用 $\nabla \times \nabla \boldsymbol{v} = 0$ 可得:

$$
\nabla \times \boldsymbol{e}
= \frac{1}{2} \nabla \times (\boldsymbol{a} \times \boldsymbol{g})
= \frac{1}{2} [(\nabla \boldsymbol{a})^T - (\nabla \cdot \boldsymbol{a}) \boldsymbol{g}]
$$

再利用 $\nabla \cdot (\nabla \times \boldsymbol{v}) = 0$ 可得:

\begin{gather*}
    \nabla \times \boldsymbol{e}
    = \frac{1}{2} (\nabla \boldsymbol{a})^T \\
    \Rightarrow
    (\nabla \times \boldsymbol{e})^T
    = \frac{1}{2} \nabla \boldsymbol{a}
\end{gather*}

然后利用 $\nabla \times \nabla \boldsymbol{a} = 0$ 便可得:

$$
\nabla \times (\nabla \times \boldsymbol{e})^T = 0
$$

上式也可以拆成下面的分量式:

$$
\nabla_\alpha \nabla_\beta e_{ij} + \nabla_i \nabla_j e_{\alpha \beta} - \nabla_\alpha \nabla_i e_{\beta j} - \nabla_\beta \nabla_j e_{\alpha i} = 0
$$

对于 $\mathring{\boldsymbol{\varepsilon}}, \boldsymbol{\varepsilon}$ ,在小变形条件下也可以求出类似的式子。

\newpage

\part{动力学定律与基本模型}

本部分先采用最一般的张量形式建立连续介质的动力学方程,然后介绍两种基本的连续介质模型。连续介质的动力学方程是物理学五大基本定律在连续介质情况下的表述,在相应前提条件下具有普适性。本书主要考虑单独一种介质的情况,也初步引导读者思考混合物的扩散和化学反应的影响,但不考虑介质与电磁场的相互作用。关于应力张量和热力学方程的论述是本章的重点。

\section{质量守恒定律}

本节讨论分别在拉格朗日观点和欧拉观点下表述质量守恒定律的细节。在拉格朗日观点下通常选取物质体为研究对象,在欧拉观点下通常考虑控制体,但也可以考虑物质体。

\subsection{欧拉观点}

在欧拉观点下,介质的密度 $\rho$ ,速度 $\boldsymbol{v}$ 等量都是选定的空间坐标 $x^i$和时间 $t$ 的函数。在连续介质中任意选取一个控制体 $V$ 并研究其中介质的质量随时间的变化。根据质量守恒定律,该质量对时间的导数与单位时间内从控制面 $\Sigma$ 离开或进入控制体的质量是平衡的,即:

$$
\frac{d}{d t} \int_V \rho d V
= - \int_\Sigma (\boldsymbol{n} \cdot \boldsymbol{v} - \boldsymbol{n} \cdot \boldsymbol{D}) \rho \, d \Sigma
$$

这里的 $\boldsymbol{n}$ 是控制面 $\Sigma$ 上的单位外法向矢量, $\boldsymbol{D}$ 是控制面 $\Sigma$ 的移动速度。需要强调,等式左边的求导运算是对 $t$ 求普通的导数,而不是求偏导数或物质导数,因为积分本身只与时间有关而与坐标无关。

如果用 $V(t)$ 表示时刻 $t$ 的控制体,其表面为 $\Sigma(t)$ ,则根据导数的定义:

\begin{gather*}
    \frac{d}{d t} \int_{V(t)} \rho(x^i, t) d V
    = \lim_{\Delta t \to 0} \frac{1}{\Delta t} \left[ \int_{V(t + \Delta t)} \rho(x^i, t + \Delta t) d V - \int_{V(t)} \rho(x^i, t) d V \right] \\
    = \lim_{\Delta t \to 0} \frac{1}{\Delta t} \left[ \int_{V(t + \Delta t)} \rho(x^i, t + \Delta t) d V - \int_{V(t + \Delta t)} \rho(x^i, t) d V \right] + \lim_{\Delta t \to 0} \frac{1}{\Delta t} \left[ \int_{V(t + \Delta t)} \rho(x^i, t) d V - \int_{V(t)} \rho(x^i, t) d V \right]
\end{gather*}

上式的前者显然给出:

$$
\int_{V(t)} \frac{\partial \rho}{\partial t} d V
$$

而后者是在区域 $V(t + \Delta t) - V(t)$ 上的积分,该区域在 $\Delta t$ 很小时变为在控制面 $\Sigma(t)$ 上构成的厚度为 $\boldsymbol{n} \cdot \boldsymbol{D}$ 的薄壳体(认为厚度有正负之分),所以相应积分化为面积分:

$$
\lim_{\Delta t \to 0} \frac{1}{\Delta t} \int_\Sigma \boldsymbol{n} \cdot \boldsymbol{D} \rho \Delta t d \Sigma
= \int_\Sigma \boldsymbol{n} \cdot \boldsymbol{D} \rho \, d \Sigma
$$

那么便可以得到:

$$
\int_{V} \frac{\partial \rho}{\partial t} d V
= - \int_\Sigma \boldsymbol{n} \cdot \boldsymbol{v} \rho d \Sigma
= - \int_{V} \nabla \cdot (\boldsymbol{v} \rho) d V
$$

即:

$$
\int_{V} \left[ \frac{\partial \rho}{\partial t} + \nabla \cdot (\boldsymbol{v} \rho) \right] d V = 0
$$

对于任意选取的物质体 $V$ ,由于物质体由同样一些物质点组成并与介质一起运动,其质量在运动过程中始终保持不变,即:

$$
\frac{d}{dt} \int_V \rho d V = 0
$$

重复与上面类似的计算过程,只是这时的薄壳体具有厚度 $\boldsymbol{n} \cdot \boldsymbol{v} \Delta t$ ,结果也是:

$$
\int_{V} \left[ \frac{\partial \rho}{\partial t} + \nabla \cdot (\boldsymbol{v} \rho) \right] d V = 0
$$

\begin{remark}
    \label{remark:suiTiJiFenQiuDao}
    类似地,对任意张量 $\boldsymbol{T}$ ,有:

    $$
    \frac{d}{dt} \int_V \boldsymbol{T} d V
    = \int_{V} [\frac{\partial \boldsymbol{T}}{\partial t} + \nabla \cdot (\boldsymbol{v} \boldsymbol{T})] d V
    $$

    这个关系式称为随体积分对时间的求导公式。
\end{remark}

如果把被积表达式中的散度项展开,再利用物质导数公式,也可以把得到的结果改写为另一种形式:

$$
\int_{V} \left( \frac{d \rho}{d t} + \rho \; \nabla \cdot \boldsymbol{v} \right) d V = 0
$$

由 $V$ 的任意性可知,被积表达式在运动连续的情况下恒为零,即:

$$
\frac{\partial \rho}{\partial t} + \nabla \cdot (\boldsymbol{v} \rho) = 0
$$

$$
\frac{d \rho}{d t} + \rho \; \nabla \cdot \boldsymbol{v} = 0
$$

来自质量守恒定律的这样的偏微分方程称为连续性方程。它还可以写为:

$$
\nabla \cdot \boldsymbol{v} = \rho \frac{d}{d t} \frac{1}{\rho}
$$

连续性方程的这些形式从不同角度刻画了质量守恒定律。方程 $\frac{\partial \rho}{\partial t} + \nabla \cdot (\boldsymbol{v} \rho) = 0$ 通常称为守恒形式的连续性方程,因为它表明,单位体积控制体内的质量对时间的变化率等于单位时间内从相应控制面进入控制体的质量。另一方面,方程 $\frac{d \rho}{d t} + \rho \; \nabla \cdot \boldsymbol{v} = 0$ 和 $\nabla \cdot \boldsymbol{v} = \rho \frac{d}{d t} \frac{1}{\rho}$ 告诉我们,物质体微元体积的相对变化率(这是速度散度的运动学意义)等于其密度的相对变化率的相反数,即其质量体积 $\frac{1}{\rho}$ 的相对变化率。

连续性方程是描述连续介质运动的基本方程之一。一般而言,任何在运动过程中保持不变的量,其体积密度 $f$ (可以是张量)都满足同样形式的连续性方程,例如:

$$
\frac{\partial f}{\partial t} + \nabla \cdot (\boldsymbol{v} f) = 0
$$

对于不可压缩介质,物质体微元的体积在运动过程中保持不变,物质体微元的密度因而也保持不变,所以连续性方程变为:

$$
\nabla \cdot \boldsymbol{v} = 0
\quad 或 \quad
\frac{d \rho}{d t} = 0
$$

当然,不可压缩介质只是实际介质的一种近似模型,是否考虑可压缩性要视具体情况而定,有时即使物质体体积变化很小也不能完全忽略这种变化的影响。

\subsection{拉格朗日观点}

在拉格朗日观点下,表征介质状态的物理量是拉格朗日坐标 $\xi^i$ 和时间 $t$ 的函数。在连续介质中任意选取一个物质体 $V$ ,其质量保持不变,所以:

$$
\frac{\partial}{\partial t} \int_V \rho (\xi^i, t) d V = 0
$$

这里的积分可以化为对拉格朗日坐标的三重积分,并且积分限不随时间变化。为此,我们用坐标面把积分域划分为诸多平行六面体微元,其体积显然等于

\begin{gather*}
    d V
    = \boldsymbol{e}_1 \cdot (\boldsymbol{e}_2 \times \boldsymbol{e}_3) d \xi^1 d \xi^2 d \xi^3
    = \varepsilon_{123} d \xi^1 d \xi^2 d \xi^3 \\
    (\text{认为} d \xi^i >0 \text{,坐标系是右手系})
\end{gather*}

三重积分对时间的导数等于其被积函数对时间的偏导数的三重积分。在之前我们已经得到公式 $\frac{\partial \varepsilon_{ijk}}{\partial t} = (\nabla \cdot \boldsymbol{v}) \varepsilon_{ijk}$ 。所以最终可得:

$$
\int_{V} \left( \frac{\partial \rho}{\partial t} + \rho \; \nabla \cdot \boldsymbol{v} \right) d V = 0
$$

根据速度散度的运动学意义,也可以直接得到上式。

在运动连续的情况下,有:

$$
\frac{\partial \rho}{\partial t} + \rho \; \nabla \cdot \boldsymbol{v} = 0
$$

这是拉格朗日变量下的连续性方程。利用欧拉坐标与拉格朗日坐标之间的变换,也可以写出上式。

如果像第一部分那样比较初始构形与瞬时构形,还可以写出其他形式的连续性方程。设初始构形下的一个物质体微元具有体积 $d \mathring{V}$ 和密度 $\mathring{\rho}$ ,它在瞬时构形下具有体积 $d V$ 和密度 $\rho$ ,则质量守恒定律表明:

$$
\mathring{\rho} d \mathring{V}
= \rho d V
$$

利用体胀因数 $\theta$ 的公式,可以得到:

$$
\frac{\mathring{\rho}}{\rho}
= 1 + \theta
= \sqrt{\frac{g}{\mathring{g}}}
$$

这些方程都是拉格朗日变量下的连续性方程。

从以上讨论可以看出,无论采用欧拉观点还是拉格朗日观点,最终都能得到相同的结果。在研究实际问题时,可以根据需要来选取更有助于理解物理本质的或者更便于运算的观点。

\subsection{控制体方法}

对控制体写出的守恒定律是连续介质必须满足的关系式,具有普遍意义。有时可以直接利用这些积分关系式研究一些简单的问题或者进行一些理论分析,这时需要把控制面上的面积分和控制体上的体积分化为代数形式,从而得到一组代数方程或偏微分方程。这种研究方法称为控制体方法,其关键在于精确地或者近似地求出所有积分,或者忽略其中的部分积分,再利用积分中值定理处理其余积分。

\section{动量守恒定律}

\subsection{质量力和面力}

连续介质所受到的力通常分为质量力和面力,它们都是矢量。后者是导致连续介质的运动远比质点或刚体的运动复杂的关键因素。

质量力是按质量分布的长程力,例如万有引力和惯性力。地球上的重力一般被当作万有引力的特例,但它其实也包括与地球转动有关的惯性力。单位质量的物质所受到的质量力称为质量力密度,记为 $\boldsymbol{F}$ 。习惯上仍把质量力密度简称为质量力。对于重力, $\boldsymbol{F}$ 就是重力加速度 $\boldsymbol{g}$ 。

面力是按面积分布的力,即曲面两侧的连续介质在该曲面上的相互作用力。单位面积上的面力(即面力强度)称为应力,需要借助于比矢量更复杂的二阶张量才能描述。本节的基本任务之一,就是引人这个二阶张量——应力张量。

设面微元 $d \sigma$ 具有单位法向矢量 $\boldsymbol{n}$ ,我们用 $\boldsymbol{p}_n$ 表示 $\boldsymbol{n}$ 所指的一侧介质作用在该面微元上的应力,则这一侧介质作用在该面微元上的面力为 $\boldsymbol{p}_n d \sigma$ 。

应力可以分解为法向应力与切向应力之和,即

$$
\boldsymbol{p}_n
= p_{nn} \boldsymbol{n} + p_{n \tau} \boldsymbol{\tau}
$$

其中 $\boldsymbol{\tau}$ 是相应面微元的单位切向矢量。

在静止流体中只有法向应力而没有切向应力。这个性质可以作为流体的定义:在静止状态下不能承受切向应力的介质称为流体。实验表明(帕斯卡定律),在静止流体中,法向应力的大小只与相应面微元的位置有关,而与其法线方向无关。静止流体中的这个法向应力通常表现为压力而非拉力,即该应力与法线方向相反,于是,静止流体中的应力可以写为:

$$
\boldsymbol{p}_n = - p \boldsymbol{n}
$$

其中 $p$ 通常称为压强(压力强度之意)。压强的概念源自静止流体,运动流体中的压强的含义将在后文中加以明确。

\subsection{积分形式的动量方程}

对于物质体:

$$
\frac{d}{dt} \int_V \rho \boldsymbol{v} d V
= \int_V \boldsymbol{F} \rho d V + \int_{\Sigma} \boldsymbol{p}_n d \Sigma
$$

而

$$
\frac{d}{d t} \int_V \rho \boldsymbol{v} d V
= \int_V \left[ \frac{\partial}{\partial t} (\rho \boldsymbol{v}) + \nabla \cdot (\rho \boldsymbol{v} \boldsymbol{v}) \right] d V
$$

\begin{remark}
    上式参见 \hyperref[remark:suiTiJiFenQiuDao]{随体积分对时间的求导公式}
\end{remark}

整理得:

$$
\frac{d}{d t} \int_V \rho \boldsymbol{v} d V
= \int_V \frac{d \boldsymbol{v}}{d t} \rho d V + \int_V [\frac{d \rho}{d t} + \rho (\nabla \cdot \boldsymbol{v})] d V
$$

代入质量守恒方程,就能得到:

$$
\int_V \frac{d \boldsymbol{v}}{d t} \rho d V
= \int_V \boldsymbol{F} \rho d V + \int_{\Sigma} \boldsymbol{p}_n d \Sigma
$$

\begin{remark}
    事实上,对于任意张量 $\boldsymbol{T}$ ,我们有:

    $$
    \frac{d}{d t} \int_V \rho \boldsymbol{T} d V
    = \int_V \rho \frac{d \boldsymbol{T}}{d t} d V
    $$

    推导方法同上
\end{remark}

对于控制体,控制体内动量的变化率等于动量的净流入率加上连续介质所受的合外力,故:

$$
\frac{d}{dt} \int_V \rho \boldsymbol{v} d V
= \int_V \boldsymbol{F} \rho d V + \int_{\Sigma} \boldsymbol{p}_n d \Sigma + \int_\Sigma \boldsymbol{n} \cdot (\boldsymbol{D} - \boldsymbol{v}) \rho \boldsymbol{v} d \Sigma
$$

类似上一节的分析方法,有:

$$
\frac{d}{dt} \int_V \rho \boldsymbol{v} d V
= \int_V \frac{\partial}{\partial t} (\rho \boldsymbol{v}) d V + \int_\Sigma \boldsymbol{n} \cdot \boldsymbol{D} \rho \boldsymbol{v} d \Sigma
$$

于是也可以得到:

$$
\int_V \frac{d \boldsymbol{v}}{d t} \rho d V
= \int_V \boldsymbol{F} \rho d V + \int_{\Sigma} \boldsymbol{p}_n d \Sigma
$$

\subsection{应力的性质}

应力的下述性质是由动量定律决定的,可以从积分形式的动量方程出发,在运动连续的前提下推导出来。这些性质对于静止的或运动的介质均成立。

首先,对于介质内部的任何一个面微元 $\Sigma$ ,其两侧介质彼此之间作用在该面微元上的面力总是大小相同,方向相反,即:

$$
\boldsymbol{p}_n = - \boldsymbol{p}_{-n}
$$

\begin{proof}
    以面微元 $\Sigma$ 为公共边界面,在其两侧各取一静止控制体微元 $V_1, V_2$ ,设它们的表面分别为 $\Sigma + \Sigma_1, \Sigma + \Sigma_2$ ,则对控制体 $V_1, V_2$ 和 $V_1 + V_2$ 分别写出动量方程,并把前两者相加再减去后者,消去彼此抵消的各项后得到:

    $$
    \int_\Sigma (\boldsymbol{p}_n + \boldsymbol{p}_{-n})d \Sigma = 0
    $$

    由 $\Sigma$ 的任意性可知上式成立
\end{proof}

其次,应力可以通过一个二阶张量 $\boldsymbol{P}$ 表示为:

$$
\boldsymbol{p}_n
= \boldsymbol{n} \cdot \boldsymbol{P}
$$

该张量称为应力张量。

\begin{remark}
    在文献中也经常把应力表示为 $\boldsymbol{p}_n = \boldsymbol{P} \cdot \boldsymbol{n}$ 。这样定义出来的应力张量与前者互为转置,如果应力张量对称(在经典情况下这是动量矩定律的推论),两者没有差别。
\end{remark}

为了证明这个结论,我们建立曲线坐标系 $x^i$ ,在某一点 $M$ 处以协变基矢量 $\boldsymbol{e}_i$ 所在直线为边建立四面体 $MABC$ ,面 $ABC$ 的单位外法向矢量为 $\boldsymbol{n}$ 。点 $M$ 到面 $ABC$ 的距离为 $h$ 。用 $S, S_1, S_2, S_3$ 表示面 $ABC, MBC, MAC, MAB$ 及其面积。

显然, $S_\alpha$ 上的单位外法向矢量可以通过逆变基矢量表示为 $- \frac{\boldsymbol{e}^\alpha}{|\boldsymbol{e}^\alpha|}$ 。 记 $\boldsymbol{p}_\alpha$ 为法线指向逆变基矢量 $\boldsymbol{e}^\alpha$ 方向的面微元上的应力。

对所取控制体写出动量方程,利用中值定理把其中的积分写为被积函数在相应积分域上的某个值(用下标*表示)与该积分域的体积或面积相乘的形式,再利用性质 $\boldsymbol{p}_n = - \boldsymbol{p}_{-n}$ ,便可得到:

$$
\frac{S h}{3} (\rho \frac{d \boldsymbol{v}}{d t} - \rho \boldsymbol{F})_*
= \boldsymbol{p}_{n*} S - \sum_\alpha \boldsymbol{p}_{\alpha*} S_\alpha
$$


面积 $S$ 与 $S_\alpha$ 之间的关系由等式

$$
\oint_\Sigma \boldsymbol{n} d \Sigma = 0
$$

给出,它是奥-高定理的推论。由之可得:

$$
\boldsymbol{n} S
= \sum_\alpha \frac{\boldsymbol{e}^\alpha}{|\boldsymbol{e}^\alpha|} S_\alpha
$$

$$
\Rightarrow
S_\beta = \boldsymbol{n} \cdot \boldsymbol{e}_\beta |\boldsymbol{e}^\beta| S
\quad (对 \beta 不求和)
$$

把这个结果代入:

$$
\frac{h}{3} (\rho \frac{d \boldsymbol{v}}{d t} - \rho \boldsymbol{F})_*
= \boldsymbol{p}_{n*} - \boldsymbol{n} \cdot \sum_\alpha |\boldsymbol{e}^\alpha| \boldsymbol{e}_\alpha \boldsymbol{p}_{\alpha*}
$$

在 $h \to 0$ 时便可得到:

$$
\boldsymbol{p}_n
= \boldsymbol{n} \cdot \sum_\alpha |\boldsymbol{e}^\alpha| \boldsymbol{e}_\alpha \boldsymbol{p}_\alpha
$$

因此,应力矢量可以表示为单位法向矢量 $\boldsymbol{n}$ 与某个量的点积。根据商法则,这个量是一个二阶张量,即应力张量 $\boldsymbol{P}$ 。

上式表明,只要知道点 $M$ 处任意三个互不相同的面微元上的应力 $\boldsymbol{p}_i$ ,就能知道这里任意方位的面微元上的应力。

\begin{remark}
    在关于应力性质的以上推导过程中,运动连续假设非常重要。
\end{remark}

\subsection{应力张量}

根据以上结果,应力张量

$$
\boldsymbol{P}
= \sum_i |\boldsymbol{e}^i| \boldsymbol{e}_i \boldsymbol{p}_i
$$

如果已知这个张量,即如果已知

$$
\boldsymbol{P}
= p^{ij} \boldsymbol{e}_i \boldsymbol{e}_j
$$

那么给定点处的应力状态完全取决于应力张量,而只要知道通过该点的三个彼此不同的面微元上的应力,就可以完全确定应力张量。

对比上面的两个式子可知,应力张量的逆变分量 $p^{ij}$ 等于法线指向逆变基矢量 $\boldsymbol{e}^i$ 的面微元上的应力的第 $j$ 个逆变分量与 $|\boldsymbol{e}^i|$ 之积。特别地,在直角坐标系下, $p^{ij}$ 是法线指向 $x^i$ 坐标轴的面微元上的应力在 $x^j$ 坐标轴上的投影。

例如,在静止流体中只有法向应力而无切向应力,所以 $\boldsymbol{p}_n = - p \boldsymbol{n}$ ,其中 $p$ 为压强,这时的应力张量 $\boldsymbol{P} = - p \boldsymbol{g}$ 是球张量。在理想流体中忽略切向应力,其应力张量同静止流体情况一样。

需要强调,应力张量在一般情况下是非对称的,在经典情况下(即在不考虑与电磁场有关的效应时),应力张量的对称性是动量矩守恒定律的推论(见下一节)。本书只考虑应力张量对称的情况。

利用应力张量很容易研究介质内一点的应力状态。例如,某面微元上的法向应力和切向应力分别为:

$$
p_{nn}
= \boldsymbol{p}_n \cdot \boldsymbol{n}
= \boldsymbol{n} \cdot \boldsymbol{P} \cdot \boldsymbol{n}
= p^{ij} n_i n_j
$$

$$
p_{n \tau}
= \sqrt{\boldsymbol{p}_n \cdot \boldsymbol{p}_n - p_{nn}^2}
= \sqrt{p^{ij} p^{kl} g_{jl} n_i n_k - (p^{ij} n_i n_j)^2}
$$

要求这两个表达式的极值,可以使用应力张量主轴坐标系以简化运算。设在此系中 $\boldsymbol{P} = \sum_i P_i \boldsymbol{e}_i \boldsymbol{e}_i$ ,则:

$$
p_{nn}
= \sum_i P_i n_i^2
$$

$$
p_{n \tau}
= \sqrt{\sum_{i < j} (P_i - P_j)^2 (n_i n_j)^2} \\
(\boldsymbol{n} \cdot \boldsymbol{n} = \sum_i n_i^2 = 1)
$$

此外还有

$$
|\boldsymbol{p}_n|
= \sqrt{\sum_i P_i^2 n_i^2}
$$

对于法向应力,三个应力张量主分量(称为主应力)都是法向应力的局部极值(它们显然也是应力的局部极值),而主方向就是相应作用面的法线方向。

对于切向应力,局部极大切向应力作用面的法线方向是应力张量两个主方向的对角线(共有三对这样的作用面,它们都是主坐标面的对角面),而切向应力极大值本身等于相应主应力之差的一半。

因此,给定点处的最大和最小的应力(法向应力)分别等于最大和最小主应力,最大切向应力等于最大和最小主应力之差的一半,最小切向应力显然为零。这些结果有助于分析材料最容易在哪里遭到破坏。

还可以计算给定点处的平均法向应力 $p^*_{nn}$ ,即极限:

$$
\lim_{a \to 0} \frac{1}{4 \pi a^2} \int_S p_{nn} d \Sigma
$$

其中积分域 $\Sigma$ 为以给定点为球心的球面, $a$ 为其半径。

由上式:

$$
p^*_{nn}
= \lim_{a \to 0} \frac{1}{4 \pi a^2} \int_S \boldsymbol{n} 
\cdot \boldsymbol{P} \cdot \boldsymbol{n} d \Sigma
= \lim_{a \to 0} \frac{1}{4 \pi a^2} \int_V \nabla \cdot (\boldsymbol{P} \cdot \frac{\boldsymbol{r}}{a}) d V
$$

$$
\Rightarrow
p^*_{nn}
= \lim_{a \to 0} \frac{1}{4 \pi a^2} \int_V \nabla_i (\frac{P^{i \cdot}_{\cdot j} r^j}{a}) d V
= \lim_{a \to 0} \frac{1}{4 \pi a^3} \int_V P^{i \cdot}_{\cdot j} \delta_i^j d V \\
= \lim_{a \to 0} \frac{1}{4 \pi a^3} P^{i \cdot}_{\cdot i} \frac{4 \pi a^3}{3}
= \frac{1}{3} p^{i \cdot}_{\cdot i}
$$

于是得到:

$$
p^*_{nn}
= \frac{1}{3} p^{i \cdot}_{\cdot i}
= \frac{1}{3} tr \boldsymbol{P}
$$

故平均法向应力就是平均主应力。

\subsection{微分形式的动量方程}

引入应力张量之后,应力沿封闭曲面的积分就可以根据奥高定理化为体积分的形式,所以动量方程化为:

$$
\int_V \frac{d \boldsymbol{v}}{d t} \rho d V
= \int_V \boldsymbol{F} \rho d V + \int_V \nabla \cdot \boldsymbol{P} \, d V
$$

在连续运动的情况下(应力张量也应当连续、可微),这个方程等价于以下微分方程:

$$
\rho \frac{d \boldsymbol{v}}{d t}
= \rho \boldsymbol{F} + \nabla \cdot \boldsymbol{P}
$$

与理论力学中的动量方程相比,这个方程多出一个散度项,而这正是连续介质力学的困难所在。

当介质静止不动时,由此可得静力学方程:

$$
\rho \boldsymbol{F} + \nabla \cdot \boldsymbol{P} = 0
$$

这个方程是求解静力学问题的基本方程。例如,对于静止流体,散度项显然可以化为梯度形式,于是:

$$
\rho \boldsymbol{F} - \nabla p = 0
$$

这个方程给出质量力作用下的静止流体中的压强所应满足的微分方程。当然,由此还可以看出,流体要想在给定质量力下保持平衡,质量力也必须满足条件:

$$
\nabla \times (\rho \boldsymbol{F}) = 0
$$

\subsection{动能定理}

从动量方程出发,可以得到介质的动能变化与各种力做功之间的关系,取动量方程与速度的点积,即可得到动能方程:

$$
\rho \frac{d}{dt} \frac{v^2}{2}
= \rho \boldsymbol{F} \cdot \boldsymbol{v} + \mathrm{div} \boldsymbol{P} \cdot \boldsymbol{v}
$$

右边第一项显然是单位体积介质所受到的质量力的功率,为了解释右边第二项的物理意义,我们把它写为散度项与另一项之差的形式,则有:

$$
\mathrm{div} \boldsymbol{P} \cdot \boldsymbol{v}
= \nabla \cdot (\boldsymbol{P} \cdot \boldsymbol{v}) - \boldsymbol{P} : \nabla \boldsymbol{v}
$$

在物质体 $V$ (表面为 $\Sigma$ )上积分,则:

$$
\frac{d}{d t} \int_V \rho \frac{v^2}{2} d V
= \int_V \rho \boldsymbol{F} \cdot \boldsymbol{v} d V + \int_\Sigma \boldsymbol{p}_n \cdot \boldsymbol{v} d \Sigma - \int_V \boldsymbol{P} : \nabla \boldsymbol{v} d V
$$

右边前两项显然是物质体 $V$ 所受到的质量力和外面力的功率,最后一项(带有负号)与物质体的变形有关,称为内应力或内面力的功率。如果应力张量对称,则内面力功率项中的被积函数也可以写为 $\boldsymbol{P} : \boldsymbol{e}$ ,其中 $\boldsymbol{e}$ 是应变率张量。从下文可以看出,这一项具有重要意义。

于是,物质体动能的变化率等于该物质体所受到的质量力、外面力和内面力的功率之和。这个结论称为动能定理。

\begin{remark}
    动能定理也可以表述为:

    $$
    \frac{d K}{dt}
    = \frac{\delta W_{in}}{d t} + \frac{\delta W_{\Sigma}^{(e)}}{d t} + \frac{\delta W_{\Sigma}^{(i)}}{d t}
    $$
\end{remark}

我们以缓慢匀速运动的平面活塞压缩柱状气缸内的理想气体为例,来说明内面力与外面力做功的区别。气体中的压强分布显然是均匀的,而速度分布必然是线性的,所以外面力的功率(活塞推动气体的功率)为 $p S v_0$ ( $S$ 是活塞的横截面积,$v_0$ 是活塞的速度大小),而内面力的功率为 $- p S v_0$ (气体受到压缩)。这两个功率彼此抵消,所以气缸内的气体动能保持不变。值得注意的是,动能定理与能量守恒定律并不矛盾。活塞对气体所做的功虽然没有转化为气体的动能,却转化为其热力学能,从而导致其温度升高(如果气缸及活塞是绝热的)。

对于理想流体(粘度为零), $\boldsymbol{P} = - p \boldsymbol{g}$ ,于是可得:

\begin{align*}
    \frac{\delta W_{\Sigma}^{(e)}}{d t}
    &= - \int_{V} p \boldsymbol{n} \cdot \boldsymbol{v} d \Sigma \\
    \frac{\delta W_{\Sigma}^{(i)}}{d t}
    &= \int_V p \, \nabla \cdot \boldsymbol{v} d V
\end{align*}

显然 $\frac{\delta W_{\Sigma}^{(i)}}{d t}$ 项是使流体体积变化的功。

动能定理当然不是一般意义下的能量守恒定律,它仅仅给出机械能之间的平衡关系,是动量定律的推论。我们将看到,动能定理有助于表述能量守恒定律,也有助于理解具体的连续介质模型中的一些参量和系数的物理意义。

\section{动量矩守恒定律}

经典情况下的动量矩守恒定律表明,控制体 $V$ 内的连续介质动量矩的变化率等于作用于这部分连续介质的所有质量力与面力的力矩之和,再减去单位时间内进出控制体的介质所具有的动量矩,即:

$$
\frac{d}{dt} \int_V \boldsymbol{r} \times \boldsymbol{v} \rho d V
= \int_V \boldsymbol{r} \times \boldsymbol{F} \rho d V + \int_\Sigma \boldsymbol{r} \times \boldsymbol{p}_n d \Sigma - \int_\Sigma \boldsymbol{r} \times [\boldsymbol{v} - \boldsymbol{D}] \rho \boldsymbol{v} \cdot \boldsymbol{n} d \Sigma
$$

上式可化为:

$$
\int_V \frac{d}{dt} (\boldsymbol{r} \times \boldsymbol{v}) \rho d V
= \int_V \boldsymbol{r} \times \boldsymbol{F} \rho d V + \int_\Sigma \boldsymbol{r} \times \boldsymbol{p}_n d \Sigma
$$

考虑到:

$$
\frac{d}{dt} (\boldsymbol{r} \times \boldsymbol{v})
= \cancel{\boldsymbol{v} \times \boldsymbol{v}} + \boldsymbol{r} \times \frac{d \boldsymbol{v}}{d t}
$$

所以我们得到动量矩守恒方程:

$$
\int_V \boldsymbol{r} \times \frac{d \boldsymbol{v}}{d t} \rho d V
= \int_V \boldsymbol{r} \times \boldsymbol{F} \rho d V + \int_\Sigma \boldsymbol{r} \times \boldsymbol{p}_n d \Sigma
$$

这个方程在实质上给出对应力张量的约束条件:动量矩守恒定律在经典情况下要求应力张量是对称的。在运动连续的情况下,右边最后一个面积分可以化为:

$$
\int_\Sigma \boldsymbol{r} \times \boldsymbol{p}_n d \Sigma
= \int_\Sigma \boldsymbol{r} \times (n_i P^{ij} \boldsymbol{e}_j) d \Sigma
= \int_V \nabla_i (P^{ij} \boldsymbol{r} \times \boldsymbol{e}_j) d \Sigma \\
= \int_V \boldsymbol{r} \times \mathrm{div} \boldsymbol{P} d V + \int_V p^{ij} (\boldsymbol{e}_i \times \boldsymbol{e}_j) d V
$$

将此式与左边的表达式合并在一起,然后利用连续性方程化简,就能得到:

$$
\int_V p^{ij} (\boldsymbol{e}_i \times \boldsymbol{e}_j) d V = 0
$$

$$
\Rightarrow
P_{ij} = P_{ji}
$$

即应力张量是对称的。

\section{热力学定律}

\subsection{热力学基本关系式}

我们首先回顾热力学的四个基本定律。

\textbf{第零定律(热平衡定律)}:处于热平衡的系统温度一定相同。或者说,存在一个状态参数(称为温度),使得处于热平衡的系统温度相同。

\textbf{第一定律}:

在一个元过程中,系统能量 $E$ 与外力做功 $W$ 、进出系统的热量 $Q$ 的关系为:

$$
d E = \delta W + \delta Q
$$

其中

$$
E = K + U
\quad
(\text{动能} + \text{内能})
$$

再由动能定理:

$$
d K = \delta W + \delta W^{(i)}
$$

\begin{remark}
    其中 $W^{(i)}$ 为内力做功
\end{remark}

便可得:

$$
d U = \delta Q - \delta W^{(i)}
$$

\begin{remark}
    功、热与状态参量 $\mu_i$ 变化的关系可以表示如下:

    $$
    \begin{cases}
        \delta W = \sum_i W_i d \mu_i \\
        \delta Q = \sum_i Q_i d \mu_i
    \end{cases}
    $$

    一般来说独立的状态参量个数为 $2$
\end{remark}

要想进一步分析,还需要结合物质的状态方程:

$$
\begin{cases}
    p = p(T, \rho) \\
    U = U(T, \rho)
\end{cases}
$$

\begin{example}
    完全气体:

    $$
    \begin{cases}
        p = \rho R T
        \quad
        R = \frac{R_0}{M} \\
        U = m c_v (T - T_0) + U_0
    \end{cases}
    $$

    其中 $R_0$ 为普适气体常数, $M$ 为气体的摩尔质量, $c_v$ 为气体的等容热容。
\end{example}

\textbf{第二定律}:

我们知道,对于任意一个封闭的过程,有:

$$
\oint \frac{\delta Q}{T} \leq 0
$$

上式取等号当且仅当此过程可逆。

由此便可以定义状态函数熵 $S$ :

$$
\exists S, d S \stackrel{\text{可逆过程}}{=\!=} \frac{\delta Q}{T}, d S \stackrel{\text{不可逆}}{>} \frac{\delta Q}{T}
$$

熵的变化可以进一步分解为两个部分:

$$
d S = d_e S + d_i S
\quad
d_e S = \frac{\delta Q}{T}
$$

其中 $d_e S, d_i S$ 分别称为熵流、熵产生,熵产生一定不为负。容易看出,分析熵的重点在于给出熵产生的表达式。

\begin{remark}
    常热容完全气体的熵:
    
    $$
    S = S_0 + m (c_v \ln \frac{T}{T_0} + R \ln \frac{V}{V_0})
    $$
\end{remark}

\textbf{第三定律}:温度趋于 $0 K$ 时,熵独立于其他热力学参量。

\begin{remark}
    此定律确定了熵的零点
\end{remark}

从上面的定律,可以得到以下关系式:

$$
\begin{cases}
    d U = \delta Q - \delta W^{(i)} \\
    d S = d_e S + d_i S \\
    d_e S = \frac{\delta Q}{T}
\end{cases}
$$

$$
\Rightarrow
T d S
= d U + \delta W^{(i)} + T d_i S
$$

上式即热力学基本关系式

若变化过程可逆,做功形式只有体积功,则:

$$
T d S
= d U + p d V
$$

若变化过程可逆,做功形式只有线性弹性形变功,则:

$$
T d S
= d U - p^{ij} d \varepsilon_{ij}
$$

\subsection{连续介质热力学方程}

前面所介绍的热力学定律适用于均质系统。对于非均质体系,我们需要引入局部平衡假设:每一个局部都是热力学平衡的。这样我们就能定义局部的热力学状态参量。

\begin{remark}
    这个假设实际上与连续介质假设是一体的
\end{remark}

于是可以将系统的参量用积分形式表示:

$$
\begin{cases}
    U = \int_V u \rho d V \\
    S = \int_V s \rho d V \\
    K = \int_V \frac{v^2}{2} \rho d V
\end{cases}
$$

其中 $u, s$ 称为质量热力学能、质量熵,分别是单位质量物质具有的热力学能、熵。

下面来研究第一定律,对任意物质体或控制体,有:

$$
\int_V \rho \frac{d}{dt} (\frac{1}{2} v^2 + u) dV
= \int_V \rho \boldsymbol{F} \cdot \boldsymbol{v} dV + \oint_\Sigma \boldsymbol{p}_n \cdot \boldsymbol{v} d \Sigma + \int_V \rho \dot{r} dV - \oint_\Sigma q_n d \Sigma
$$

其中 $\dot{r}$ 是单位质量产热量, $q_n$ 是单位面积传热量,即传热强度。

\begin{remark}
    单位质量产热量旨在描述内热源。写出此项的前提条件是『总能量中不包含内热源所具有的能量』。运用公式时这一项应加以说明。
\end{remark}

上式减去动能定理方程即可得:

$$
\int_V \rho \frac{du}{dt} dV
= \int_V \boldsymbol{P} : \nabla \boldsymbol{v} dV + \int_V \rho \dot{r} dV - \oint_\Sigma q_n d \Sigma
$$

类似于应力张量,可以定义热流矢量 $\boldsymbol{q}$ ,使得:

$$
q_n = \boldsymbol{n} \cdot \boldsymbol{q}
$$

\begin{proof}
    为证其存在,建立曲线坐标系 $x^i$ ,在某点 $M$ 处以协变基矢量 $\boldsymbol{e}_i$ 所在直线为边建立四面体 $MABC$ ,面 $ABC$ 的单位外法向矢量为 $\boldsymbol{n}$ 。点 $M$ 到面 $ABC$ 的距离为 $h$ 。用 $S, S_1, S_2, S_3$ 表示面 $ABC, MBC, MAC, MAB$ 及其面积。

    显然, $S_\alpha$ 上的单位外法向矢量可以通过逆变基矢量表示为 $- \frac{\boldsymbol{e}^\alpha}{|\boldsymbol{e}^\alpha|}$ 。 记 $q_\alpha$ 为法线指向逆变基矢量 $\boldsymbol{e}^\alpha$ 方向的面微元上的传热强度。

    对所取控制体写出方程,利用中值定理把其中的积分写为被积函数在相应积分域上的某个值(用下标*表示)与该积分域的体积或面积相乘的形式,再利用性质 $q_n = - q_{-n}$ ,便可得到:

    $$
    \frac{S h}{3} (\rho \frac{d u}{d t} - \boldsymbol{P} : \nabla \boldsymbol{v} - \rho \dot{r})_*
    = - q_{n*} S + \sum_\alpha q_{\alpha*} S_\alpha
    $$

    又:

    $$
    \oint_\Sigma \boldsymbol{n} d \Sigma = 0
    \;
    \Rightarrow
    \;
    \boldsymbol{n} S
    = \sum_\alpha \frac{\boldsymbol{e}^\alpha}{|\boldsymbol{e}^\alpha|} S_\alpha
    $$

    $$
    \Rightarrow
    S_\beta = \boldsymbol{n} \cdot \boldsymbol{e}_\beta |\boldsymbol{e}^\beta| S
    \quad (\text{对} \beta \text{不求和})
    $$

    代入,得:

    $$
    \frac{h}{3} (\rho \frac{d u}{d t} - \boldsymbol{P} : \nabla \boldsymbol{v} - \rho \dot{r})_*
    = - q_{n*} + \boldsymbol{n} \cdot \sum_\alpha |\boldsymbol{e}^\alpha| \boldsymbol{e}_\alpha q_{\alpha*} S_\alpha
    $$

    再让 $h \to 0$ 便可得到:

    $$
    q_n
    = \boldsymbol{n} \cdot \sum_\alpha |\boldsymbol{e}^\alpha| \boldsymbol{e}_\alpha q_\alpha
    $$

    于是得到传热矢量
    
    $$
    \boldsymbol{q} = \sum_\alpha |\boldsymbol{e}^\alpha| \boldsymbol{e}_\alpha q_\alpha
    $$
\end{proof}

这样便可以得到微分形式的方程:

$$
\rho \frac{d u}{d t}
= \boldsymbol{P} : \nabla \boldsymbol{v} + \rho \dot{r} - \nabla \cdot \boldsymbol{q}
$$

\begin{remark}
    传热有三种形式:热传导、热对流、热辐射。对于热传导,我们有傅立叶热传导定律:

    $$
    \boldsymbol{q} = - \boldsymbol{\kappa} \cdot \nabla T
    $$

    其中 $\boldsymbol{\kappa}$ 称为热导率张量
\end{remark}

接下来看第二定律,有:

$$
\int_V \rho \frac{d s}{dt} d V
= \int_V \rho \frac{d_i s}{dt} d V + \int_V \rho \frac{d_e s}{dt} d V
$$

$$
\frac{d s}{dt}
= \frac{d_i s}{dt} + \frac{d_e s}{dt}
$$

要想进一步分析方程,必须从具体模型得到熵产生。

\begin{example}
    若认为系统熵变只由热传导引起,则:

    $$
    T \rho ds = \rho d q = - \nabla \cdot \boldsymbol{q} dt
    $$

    熵流总可以写成散度形式,故作如下操作:

    $$
    \rho ds
    = - \frac{1}{T} \nabla \cdot \boldsymbol{q} dt
    = - \nabla \cdot (\frac{\boldsymbol{q}}{T}) dt + \boldsymbol{q} \cdot \nabla \frac{1}{T} dt
    $$

    即可得到:

    $$
    \rho \frac{d_e s}{dt}
    = - \nabla \cdot (\frac{\boldsymbol{q}}{T})
    \qquad
    \rho \frac{d_i s}{dt}
    = \boldsymbol{q} \cdot \nabla \frac{1}{T}
    $$
\end{example}

可以证明: $\frac{d_i s}{dt} \geq 0$

\section{黏性流体模型}

\subsection{本构方程}

变形很慢时,近似于静止流体,应力张量满足:

$$
\boldsymbol{P} = - p \boldsymbol{g}
\quad
\text{或}
\quad
p^{ij} = - p g^{ij}
$$

流体运动速度较大时,还要考虑切向应力,故需添上一项:

$$
p^{ij}
= - p g^{ij} + \tau^{ij} (e_{\alpha \beta}, T, g^{\alpha \beta} \dots)
$$

$\tau^{ij}$ 称为黏性应力张量的分量

考虑一种简单的假设:$\tau^{ij}$ 满足如下的线性关系

$$
\tau^{ij} = A^{i j \alpha \beta} e_{\alpha \beta}
$$

上式中 $A^{i j \alpha \beta}$ 与 $e_{\alpha \beta}$ 无关,其所对应的张量称为黏性系数张量。实验证明这个假设对于多数流体是合理的。一般来说,黏性系数张量还是对称的、各向同性的。

\begin{remark}
    各向同性是指描述某种物理性质的张量分量在旋转变化下保持不变

    可以证明如下定理:

    \begin{itemize}
        \item 二阶各向同性张量形如
        $$
        \lambda \boldsymbol{g}
        $$
        \item 三阶各向同性张量形如
        $$
        \lambda \boldsymbol{\varepsilon}
        $$
        \item 四阶各向同性张量形如
        $$
        \lambda g^{ij} g^{\alpha \beta} + \mu (g^{i \alpha} g^{j \beta} + g^{i \beta} g^{j \alpha}) + \nu (g^{i \alpha} g^{j \beta} - g^{i \beta} g^{j \alpha})
        $$
    \end{itemize}
\end{remark}

于是可以把 $A^{i j \alpha \beta}$ 表示如下:

$$
A^{i j \alpha \beta}
= \lambda g^{ij} g^{\alpha \beta} + \mu (g^{i \alpha} g^{j \beta} + g^{i \beta} g^{j \alpha})
$$

进而有:

$$
\tau^{ij}
= \lambda g^{ij} e^{\alpha \cdot}_{\cdot \alpha} + 2 \mu e^{ij}
= \lambda g^{ij} \nabla \cdot \boldsymbol{v} + 2 \mu e^{ij}
$$

这样我们就得到了各向同性线性黏性流体的本构方程

$$
p^{ij}
= - p g^{ij}  + \lambda g^{ij} \nabla \cdot \boldsymbol{v} + 2 \mu e^{ij}
$$

这里的 $p$ 往往并不是压强,一般称为热力学压强,在\ref{sec:viscousFluid}节将解释其意义

\subsection{剪切黏度与体积黏度}

下面来研究 $\mu, \lambda$ 的物理意义。

$\mu$ 是剪切黏度,即流体承受剪应力时,剪应力与流体单位速度差的比值,且不小于零。对于简单的一维流动,有如下的数学表述:

$$
\tau = \mu \frac{d v}{d y}
$$

式中 $\tau$ 为剪应力,沿 $x$ 轴方向, $v$ 为速度场在 $x$ 方向的分量, $y$ 为与 $x$ 垂直方向的坐标。

$\lambda$ 的物理意义需要借助平均法向应力 $p^*_{nn}$ 来展现。我们知道:

$$
p^*_{nn}
= \lim_{a \to 0} \frac{1}{4 \pi a^2} \int_S \boldsymbol{n} \cdot \boldsymbol{P} \cdot \boldsymbol{n} \, d \Sigma
$$

其中积分域 $\Sigma$ 为以给定点为球心的球面, $a$ 为其半径,代入本构方程可算得:

$$
p^*_{nn}
= \frac{1}{3} p^{i \cdot}_{\cdot i}
= - p + (\lambda + \frac{2}{3} \mu) \nabla \cdot \boldsymbol{v}
$$

\begin{remark}
    若流体不可压缩,则

    $$
    p^*_{nn} = - p
    $$
\end{remark}

由上可知, $\lambda$ 的物理意义实际由 $\mu' = \lambda + \frac{2}{3} \mu$ 间接给出,我们称之为体积黏度/第二黏度

为了继续研究体积黏度,我们再来计算单位体积的内面力功率 $- \boldsymbol{P} : \nabla \boldsymbol{v}$ ,可得:

\begin{align*}
    - \boldsymbol{P} : \nabla \boldsymbol{v}
    &= - p^{ij} \nabla_i v_j \\
    &= (p g^{ij} - \lambda g^{ij} \nabla \cdot \boldsymbol{v} - 2 \mu e^{ij}) \nabla_i v_j \\
    &= \boldsymbol{P} \nabla \cdot \boldsymbol{v} - \lambda (\nabla \cdot \boldsymbol{v})^2 - 2 \mu e^{ij} e_{ij}
\end{align*}

\begin{remark}
    上式利用了 $\boldsymbol{e}$ 的对称性
\end{remark}

$$
\Rightarrow
- \boldsymbol{P} : \nabla \boldsymbol{v}
= \boldsymbol{P} \nabla \cdot \boldsymbol{v} - \mu' (\nabla \cdot \boldsymbol{v})^2 - 2 \mu (e^{ij} e_{ij} - \frac{1}{3} (\nabla \cdot \boldsymbol{v})^2)
$$

\begin{remark}
    可以证明 $e^{ij} e_{ij} - \frac{1}{3} (\nabla \cdot \boldsymbol{v})^2 \geq 0$ ,只要换入主坐标系即可求得:

    $$
    e^{ij} e_{ij} - \frac{1}{3} (\nabla \cdot \boldsymbol{v})^2
    = \frac{1}{3} \sum_{i < j} (e_i - e_j)^2
    $$

    其中 $e_i$ 是 $\boldsymbol{e}$ 的主分量
\end{remark}

定义耗散函数

$$
\varphi = \tau^{ij} e_{ij} = \mu' (\nabla \cdot \boldsymbol{v})^2 + 2 \mu (e^{ij} e_{ij} - \frac{1}{3} (\nabla \cdot \boldsymbol{v})^2)
$$

则耗散函数总不小于零(在\ref{sec:viscousFluid}节解释),且:

$$
- \boldsymbol{P} : \nabla \boldsymbol{v}
= \boldsymbol{P} \nabla \cdot \boldsymbol{v} - \varphi
$$

\subsection{Navier-Stokes 方程}

认为黏度与体积黏度是常数(虽然实际上它们与温度有关),则可由动量方程得:

$$
\rho \frac{d \boldsymbol{v}}{d t}
= \rho \boldsymbol{F} + (\mu' + \frac{1}{3} \mu) \nabla \nabla \cdot \boldsymbol{v} + \mu \Delta \boldsymbol{v} - \nabla p
$$

在多数场合下这个方程精度很高,误差在 $1 \%$ 以下。不过这个方程对于气体难以应用,因为气体同时存在不可忽略的传热衰减,与第二黏性的贡献难以分开。

\begin{example}
    一维管道流动方程
    $$
    \begin{cases}
        \rho \frac{d v}{d t}
        = (\lambda + 2 \mu) \frac{\partial^2 v}{\partial x^2} - \frac{\partial p}{\partial x} \\
        \boldsymbol{v}|_{\text{管壁}}
        = \boldsymbol{v}_{\text{管壁}}
        \quad
        (\text{黏附条件})
    \end{cases}
    $$
\end{example}

当流速过高时此方程难以得到数值解。此时流体内有湍流,各物理量往往表现为平均值与脉动值的叠加,且脉动值呈明显的随机状态。

\begin{remark}
    此外,还有一种简单粗暴的假设:

    $$
    \mu' = 0
    \qquad
    (\text{Stokes假设})
    $$

    但这个假设仅对单原子气体成立。对于多原子气体,由于旋转自由度的存在,气体能量转化(微观能量再分配)更久,此时假设并不适用。
\end{remark}

\section{线性弹性体模型}

\subsection{本构方程}

在数学形式上线性弹性体模型与黏性流体模型十分相近。线性弹性体满足胡克定律:

$$
p^{ij} = C^{i j \alpha \beta} \varepsilon_{\alpha \beta}
$$

\begin{remark}
    上式要求变形为小变形(即 $\varepsilon_{\alpha \beta} \ll 1$ )

    小变形条件下:

    $$
    \varepsilon_{\alpha \beta} = \frac{1}{2} (\nabla_{\alpha} w_{\beta} + \nabla_{\beta} w_{\alpha})
    $$
\end{remark}

上式即线性弹性体的本构方程,其中 $C^{i j \alpha \beta}$ 对应的张量称为弹性模量张量,其与变形无关。

在各项同性情况下,可进一步得到:

$$
p^{ij}
= \lambda g^{ij} \varepsilon^{\alpha \cdot}_{\cdot \alpha} + 2 \mu \varepsilon^{ij}
$$

上式中的 $\lambda, \mu$ 称为拉梅系数。

\subsection{拉梅系数 $\;$ 拉梅方程}

两个拉梅系数的物理意义也和黏性流体模型相似。 $\mu$ 是剪切模量,一定大于零。 $\lambda$ 借助平均法向应力 $p^*_{nn}$ 体现物理意义:

$$
p_{nn}^*
= (\lambda + \frac{2}{3} \mu) \varepsilon^{\alpha \cdot}_{\cdot \alpha}
= (\lambda + \frac{2}{3} \mu) \nabla \cdot \boldsymbol{w}
= K \theta
$$

其中 $\theta = \nabla \cdot \boldsymbol{w}$ 是体涨系数。我们称 $K = \lambda + \frac{2}{3} \mu$ 为体积模量, $K$ 一定大于零。

再看内面力功率,为方便计算,引入二次型:

$$
X = \frac{1}{2} C^{i j \alpha \beta} \varepsilon_{ij} \varepsilon_{\alpha \beta}
$$

则:

$$
\frac{\partial X}{\partial \varepsilon_{ij}}
= p^{ij}
\qquad
\frac{\partial X}{\partial t}
= p^{ij} e_{ij}
$$

\begin{remark}
    上式利用了 $\boldsymbol{e}$ 的对称性
\end{remark}

所以内面力功率:

$$
- p^{ij} e_{ij}
= - \frac{\partial X}{\partial t}
$$

称 $X$ 为弹性势能。 $X$ 也可以写作:

\begin{align*}
    X
    &= \frac{1}{2} p^{ij} \varepsilon_{ij} \\
    &= \frac{1}{2} \lambda \theta^2 + \mu \boldsymbol{\varepsilon} : \boldsymbol{\varepsilon} \\
    &= \frac{1}{2} K \theta^2 + \mu (\boldsymbol{\varepsilon} : \boldsymbol{\varepsilon} - \frac{1}{3} \theta^2)
\end{align*}

认为两个拉梅系数是常数,则可由动量方程得:

$$
\rho \frac{d \boldsymbol{v}}{d t}
= \rho \frac{d}{dt} \frac{d \boldsymbol{w}}{d t}
\stackrel{\cdot}{=} \rho \frac{\partial^2 \boldsymbol{w}}{\partial^2 t}
= \rho \boldsymbol{F} + (K + \frac{1}{3} \mu) \nabla \nabla \cdot \boldsymbol{w} + \mu \Delta \boldsymbol{w}
$$

此方程称为拉梅方程。在此方程中可以近似取 $\rho$ 为 $\mathring{\rho}$ ,也可以取更精确的 $\mathring{\rho} (1 + \theta)$

\subsection{杨氏模量 $\;$ 泊松比}

实验确定拉梅系数的一个方法为单轴拉伸实验。应力沿一个固定的方向,记此方向为 $1$ ,则:

$$
p^{11} = E \varepsilon^{11}
$$

$E$ 称为杨氏模量,可以测出。

此外,对于另外两个方向的应变,有:

$$
\varepsilon^{22}
= \varepsilon^{33}
= - \nu \varepsilon^{11}
$$

其中 $\nu$ 为材料的泊松比。

然后,再由本构方程:

$$
\begin{cases}
    p^{11} = E \varepsilon^{11} = \lambda (\varepsilon^{11} + \varepsilon^{22} + \varepsilon^{33}) + 2 \mu \varepsilon^{11} \\
    0 = \lambda (\varepsilon^{11} + \varepsilon^{22} + \varepsilon^{33}) + 2 \mu \varepsilon^{22}
\end{cases}
$$

可得:

$$
\begin{cases}
    E = \frac{(3 \lambda + 2 \mu) \mu}{\lambda + \mu} \\
    \nu
= \frac{\lambda}{2 (\lambda + \mu)}
\end{cases}
$$

由此即可确定拉梅系数。

\begin{remark}
    对于静止的线性弹性体,往往可以如下近似:
    
    $$
    \cancel{\rho \boldsymbol{F}} + \nabla \cdot \boldsymbol{P} = 0
    $$

    而流体往往不能如此近似。
\end{remark}

\section{封闭的连续介质力学方程组}

综合前面得到的诸方程:

$$
\begin{cases}
    \frac{d \rho}{d t} + \rho \nabla \cdot \boldsymbol{v} = 0 \\
    \rho \frac{d \boldsymbol{v}}{d t}
    = \rho \boldsymbol{F} + \nabla \cdot \boldsymbol{P} \\
    \rho \frac{d u}{d t}
    = \boldsymbol{P} : \nabla \boldsymbol{v} + \rho \dot{r} - \nabla \cdot \boldsymbol{q} \\
    \frac{d s}{dt}
    = \frac{d_i s}{dt} + \frac{d_e s}{dt}
\end{cases}
$$

与物质的本构方程、熵产生方程等,便可得到封闭的连续介质力学方程组

以下举几个例子

\subsection{理想流体}

只考虑热传导,并注意到 $\boldsymbol{P} = - p \boldsymbol{g}$ ,有:

$$
\begin{cases}
    \frac{d \rho}{d t} + \rho \nabla \cdot \boldsymbol{v} = 0 \\
    \rho \frac{d \boldsymbol{v}}{d t}
    = \rho \boldsymbol{F} - \nabla p \\
    \rho \frac{d u}{d t}
    = - p \nabla \cdot \boldsymbol{v} - \nabla \cdot \boldsymbol{q} \\
    \rho \frac{ds}{dt}
    = - \frac{1}{T} \nabla \cdot \boldsymbol{q} \\
    \boldsymbol{q} = - \kappa \nabla T \\
    p = p(T, \rho) \\
    u = u(T, \rho)
\end{cases}
$$

理想流体可以当成双参量介质,这是因为由上可得:

$$
T \frac{ds}{dt}
= \frac{d u}{d t} + \frac{p}{\rho} \nabla \cdot \boldsymbol{v}
= \frac{d u}{d t} + p \frac{d}{dt} \frac{1}{\rho}
$$

即:

$$
T ds
= du + p \, d (\frac{1}{\rho})
$$

\begin{remark}
    进一步,认为 $u = c_v (T - T_0) + u_0$ ,则:

    $$
    \rho c_v \frac{d T}{d t}
    = - p \nabla \cdot \boldsymbol{v} + \kappa \Delta T
    $$

    若流体不可压缩,则上式退化为:

    $$
    \rho c_v \frac{d T}{d t}
    = \kappa \Delta T
    $$
\end{remark}

\subsection{黏性流体}
\label{sec:viscousFluid}

只考虑热传导,由前面的结果得:

$$
\begin{cases}
    \frac{d \rho}{d t} + \rho \nabla \cdot \boldsymbol{v} = 0 \\
    \rho \frac{d \boldsymbol{v}}{d t}
    = \rho \boldsymbol{F} + (\mu' + \frac{1}{3} \mu) \nabla \nabla \cdot \boldsymbol{v} + \mu \Delta \boldsymbol{v} - \nabla p \quad (\text{Navier-Stokes}) \\
    \rho \frac{d u}{d t}
    = - p \nabla \cdot \boldsymbol{v} - \nabla \cdot \boldsymbol{q} + \varphi \\
    \rho \frac{ds}{dt}
    = - \frac{1}{T} \nabla \cdot \boldsymbol{q} + \frac{\varphi}{T} \\
    \varphi = \tau^{ij} e_{ij} = \mu' (\nabla \cdot \boldsymbol{v})^2 + 2 \mu (e^{ij} e_{ij} - \frac{1}{3} (\nabla \cdot \boldsymbol{v})^2) \\
    \boldsymbol{q} = - \kappa \Delta T \\
    p = p(T, \rho) \\
    u = u(T, \rho)
\end{cases}
$$

各向同性线性黏性流体可以当成双参量介质,这是因为由上可得:

$$
T \frac{ds}{dt}
= \frac{d u}{d t} + \frac{p}{\rho} \nabla \cdot \boldsymbol{v}
= \frac{d u}{d t} + p \frac{d}{dt} \frac{1}{\rho}
$$

即:

$$
T ds
= du + p \, d (\frac{1}{\rho})
$$

上式解释了 $p$ 的物理意义:热力学压强

此外,还可以得到:

$$
\rho \frac{d_i s}{dt}
= \rho \frac{d_i s_q}{dt} + \rho \frac{d_i s_\varphi}{dt}
= \boldsymbol{q} \cdot \nabla \frac{1}{T} + \frac{\varphi}{T}
$$

这说明 $\varphi \geq 0$ ,进而说明第二黏度非负

\subsection{弹性体}

只考虑热传导,由前面的结果得:

$$
\begin{cases}
    \frac{d \rho}{d t} + \rho \nabla \cdot \boldsymbol{v} = 0 \\
    \rho \frac{\partial^2 \boldsymbol{w}}{\partial^2 t}
    = \rho \boldsymbol{F} + (K + \frac{1}{3} \mu) \nabla \nabla \cdot \boldsymbol{w} + \mu \Delta \boldsymbol{w} \\
    \rho \frac{d u}{d t}
    = p^{ij} e_{ij} - \nabla \cdot \boldsymbol{q} \\
    \rho \frac{ds}{dt}
    = - \frac{1}{T} \nabla \cdot \boldsymbol{q}  \\
    p^{ij}
    = \lambda g^{ij} \varepsilon^{\alpha \cdot}_{\cdot \alpha} + 2 \mu \varepsilon^{ij} \\
    \boldsymbol{q} = - \kappa \Delta T \\
    u = u(T, \rho)
\end{cases}
$$

由上可得:

$$
\rho T ds
= \rho du - p^{ij} e_{ij}
$$

\end{document}